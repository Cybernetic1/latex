\documentclass[16pt]{beamer}

\usepackage[CJKspace]{xeCJK}
\setCJKmainfont[BoldFont=AR PL KaitiM GB,ItalicFont=AR PL KaitiM GB]{SimHei}

%\usepackage{newtxtext,newtxmath}	% use Times Roman font
%\usefonttheme{serif}
\usefonttheme{professionalfonts}
%\setbeamertemplate{theorems}[numbered]
\setbeamertemplate{caption}{\insertcaption} 	% no `Figure' prefix before caption

\mode<presentation> {

%\usetheme{default}
%\usetheme{AnnArbor}
%\usetheme{Antibes}
%\usetheme{Bergen}
%\usetheme{Berkeley}
%\usetheme{Berlin}
%\usetheme{Boadilla}
%\usetheme{CambridgeUS}
%\usetheme{Copenhagen}
\usetheme{Darmstadt}
%\usetheme{Dresden}
%\usetheme{Frankfurt}
%\usetheme{Goettingen}
%\usetheme{Hannover}
%\usetheme{Ilmenau}
%\usetheme{JuanLesPins}
%\usetheme{Luebeck}
%\usetheme{Madrid}
%\usetheme{Malmoe}
%\usetheme{Marburg}
%\usetheme{Montpellier}
%\usetheme{PaloAlto}
%\usetheme{Pittsburgh}
%\usetheme{Rochester}
%\usetheme{Singapore}
%\usetheme{Szeged}
%\usetheme{Warsaw}

%\usecolortheme{albatross}
%\usecolortheme{beaver}
%\usecolortheme{beetle}
%\usecolortheme{crane}
%\usecolortheme{dolphin}
%\usecolortheme{dove}
%\usecolortheme{fly}
%\usecolortheme{lily}
%\usecolortheme{orchid}
%\usecolortheme{rose}
%\usecolortheme{seagull}
\usecolortheme{seahorse}
%\usecolortheme{whale}
%\usecolortheme{wolverine}

%\setbeamertemplate{footline} % To remove the footer line in all slides uncomment this line
%\setbeamertemplate{footline}[page number] % To replace the footer line in all slides with a simple slide count uncomment this line
\setbeamertemplate{navigation symbols}{} % To remove the navigation symbols from the bottom of all slides uncomment this line
}

%\setbeamertemplate{headline}{}
\setbeamersize{text margin left=1mm,text margin right=1mm} 
\settowidth{\leftmargini}{\usebeamertemplate{itemize item}}
\addtolength{\leftmargini}{\labelsep}

\usepackage[backend=biber,bibstyle=authoryear,citestyle=../authoryearbrack]{biblatex}
\bibliography{../AGI-book}
\renewcommand*{\bibfont}{\footnotesize}

\usepackage{graphicx} % Allows including images
\usepackage{verbatim} % comments
% \usepackage{tikz-cd}  % commutative diagrams
% \newcommand{\tikzmark}[1]{\tikz[overlay,remember picture] \node (#1) {};}
% \usepackage{booktabs} % Allows the use of \toprule, \midrule and \bottomrule in tables
% \usepackage{amssymb}  % \leftrightharpoons
\usepackage{wasysym} % frownie face
\usepackage{newtxtext,newtxmath}	% Times New Roman font

\newcommand{\red}[1]{{\color{red}#1}}
\newcommand{\vect}[1]{\boldsymbol{#1}}
\newcommand*\confoundFace{$\vcenter{\hbox{\includegraphics[scale=0.2]{../confounded-face.jpg}}}$}
\renewcommand{\smiley}{$\vcenter{\hbox{\includegraphics[scale=0.05]{../smiling-face.png}}}$}

\makeatletter
\renewcommand{\boxed}[1]{\fbox{\m@th$\displaystyle\scalebox{0.9}{#1}$} \,}
\makeatother
\newif\ifframeinlbf
\frameinlbftrue
\makeatletter
\newcommand\listofframes{\@starttoc{lbf}}
\makeatother

\addtobeamertemplate{frametitle}{}{%
	\ifframeinlbf
	\addcontentsline{lbf}{section}{\protect\makebox[2em][l]{%
			\protect\usebeamercolor[fg]{structure}\insertframenumber\hfill}%
		\insertframetitle\par}%
	\else\fi
}

%----------------------------------------------------------------------------------------
%	TITLE PAGE
%----------------------------------------------------------------------------------------

\title[HK neutral party]{\Huge《香港中立》}

\author{HK.neutrality@gmail.com}
%\author{\cc{YKY 甄景贤}{YKY}} % Your name
%\institute[] % Your institution as it will appear on the bottom of every slide, may be shorthand to save space
%{
%Independent researcher, Hong Kong \\ % Your institution for the title page
%\medskip
%\textit{generic.intelligence@gmail.com} % Your email address
%}
\date{\today} % Date, can be changed to a custom date

\begin{document}

\frameinlbffalse
\frame{\titlepage}

\begin{frame}
\frametitle{Table of contents}
\listofframes
\vspace*{0.5cm}
Hello 各位朋友 \smiley

近来有些进展,但只能算是「中段结果」,和大家分享一下。 \\
也希望能找到合作者。
\end{frame}

%---------------- this is for when you're using \part's ----------------------------------
%\begin{frame}
%\frametitle{Summary}
%
%{\usebeamerfont*{frametitle} Part I %\usebeamercolor[fg]{frametitle}
% ~ ~ ~ Deep reinforcement learning}
%%\tableofcontents[part=1]
%
%\vspace{1.5cm}
%{\usebeamerfont*{frametitle} Part II %\usebeamercolor[fg]{frametitle}
% ~ ~ ~ Logical structure}
%%\tableofcontents[part=2]
%\end{frame}

%----------------------------------------------------------------------------------------
%	PRESENTATION SLIDES
%----------------------------------------------------------------------------------------

%------------------------------------------------

%\part{title}

%\section[Section]{}
%\frame{\sectionpage}
\frameinlbftrue
\begin{frame}
\frametitle{去除民主选举制}
\begin{itemize}
	\item 民主 令社会不断 回归到平均值,民主 不适合 人口多、发展中的国家,选举 is a waste of time
	
	\item {\color{red}[rephrase]} 民主是西方欺压其他民族的手段
	
	\item 西方历史上,民主的发源地 希腊雅典,亦没有因为有民主而免於战败。 民主的 Athens 被 Sparta 打败,即著名的 Pelopponesian War
	
	\item 柏拉图、亚里士多德 等哲学家 提出 政治的 \textbf{循环} (cycle): \\
	mob rule $\Rightarrow$ monarchy $\Rightarrow$ tyranny $\Rightarrow$ aristocracy $\Rightarrow$ oligarchy $\Rightarrow$ democracy $\Rightarrow$ mob rule a
	
	\item 「中国模式」令中国近10-20年经济起飞,必然做对了某些事。 其实 这证明了民主对於经济进步是不必要的。 印度有民主,但经济一样起唔到飞
	
	\item 中国在不平等条约下割让香港,所以中国没有义务 honor 中英联合声明,特别是 普选制度。 但在精神上香港中立派 传承了一国两制的安定繁荣 目标
	
\end{itemize}
\end{frame}

\begin{frame}
\frametitle{经济自由化 \textbullet 去除家长式管治}
\begin{itemize}
	\item 只有自由市场、经济竞争,才能令 社会进步
	\item 香港要放弃「家长式」管治 (paternalistic governance),政府要「放手」(\textbf{deregulation}) 让人们 自己 解决问题
	\item 亚洲 流行 家长式 管治,原因是 很多亚洲人像 \textbf{巨婴}。 左倾 和 民主 贪得无厌的诉求,都是基於 不劳而获的心态
	\item 美好的生活 $\supseteq$ 有赚钱的自由; 美国梦 (American dream) 指的是 美国人不论出身,都有可能透过努力成为富人
	\item Deregulation 的做法可以是: 企业联合集资,去除政府管制,并在过渡期获得制度上补偿
	\item 在香港进行 \textbf{土地改革} 不是没有可能的,但土地的经济学和一般财产可能有不同 \cite{Ryan-Collins2017} \cite{Farvacque-Vitkoviac1992} \cite{Blomley2004} \cite{Linklater2013} \cite{Adams2015},而且涉及「地产霸权」的利益
\end{itemize}
\end{frame}

\begin{frame}[plain]
% \frametitle{交换不变的 神经网络}
\begin{itemize}
	% \item 就算考虑 表示论 (representation theory),所有 Abel 群的不可约复表示 都是 1-维的。 $F_n^{\text{Ab}}$ 的表示,恰好是 $n$ 个 1-维 表示的直和。 没卵用!
	\item 上页结论是: $\mathbb{Z}^n$ 不可能嵌入到更低维的空间内,除非使用某种 \red{fractal} 方法。  然而,fractals 恰好是 神经网络 这件武器的「射程范围」之外! 
	\item 换个想法,通过 类似 weights-sharing 的 \textbf{约束},令 神经网络 变成 permutation invariant (= \red{Symmetric NN})
	\item 这方法 必需设 activation function = polynomial
	\item 缺点是: 约束的数量 随著 层数 增加而 指数式增长,暂时只能做 1-2 层的,每层 = 2次多项式
	\item 优点是: 反正我们希望 神经网络 是 sparse(减少权重空间的自由度),而这个做法同时具有逻辑 bias
	\item 细节我会在另一篇论文讲述
\end{itemize}
\printbibliography
\begin{center}
	多谢收看 \smiley
\end{center}
\end{frame}

\begin{comment}

\begin{frame}
\frametitle{References}
\footnotesize{
\begin{thebibliography}{99} % Beamer does not support BibTeX so references must be inserted manually as below
\bibitem[]{} Bart Jacobs (1999)
\newblock Categorical logic and type theory
% \newblock \emph{North Holland, Studies in logic} v141.

\bibitem[]{} Robert Goldblatt (2006)
\newblock Topoi -- the categorical analysis of logic

\end{thebibliography}
}
\end{frame}

\end{comment}

\end{document} 