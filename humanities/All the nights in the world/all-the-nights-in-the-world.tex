\documentclass[a6paper]{article}
\usepackage[margin=5mm]{geometry}
\usepackage{tgheros}
\usepackage[T1]{fontenc}

\usepackage[most]{tcolorbox}
\usepackage{ulem}
\usepackage[sf,bf,big,raggedright,compact]{titlesec}	% change section color to blue
% \usepackage[backend=biber,bibstyle=authoryear,citestyle=../authoryearbrack]{biblatex}

\ifdefined\chinchin
\usepackage[CJKspace]{xeCJK}
\setCJKmainfont[BoldFont=SimHei,ItalicFont=AR PL KaitiM GB]{KaiTi}
\newcommand{\cc}[2]{#1}
\else
\newcommand{\cc}[2]{#2}
\fi

%\setlength{\headheight}{0cm}
%\setlength{\hoffset}{0cm}
%\setlength{\topmargin}{-2cm}
%\setlength{\oddsidemargin}{-2cm}
%\setlength{\evensidemargin}{-2cm}
%\setlength{\textwidth}{20cm}
%\setlength{\textheight}{26cm}
%\setlength{\headsep}{0cm}
%\setlength{\topskip}{0cm}
%\setlength{\footskip}{0.9cm}  % between bottom of page and page number
%\setlength{\floatsep}{0cm}
%\setlength{\textfloatsep}{0.6cm}
%\setlength{\intextsep}{0.5cm}
%\setlength{\parindent}{0em}   % em = width of capital M

\setcounter{secnumdepth}{2}		% no section numbers

\titleformat{\section}[hang]{\LARGE}{}{0pt}{}
% \titleformat{\subsection}[hang]{\bfseries\large\color{blue}}{\thesubsection \hspace{5pt}}{0pt}{}

\renewcommand{\labelitemi}{$\textbullet$}

\setlength{\parindent}{0pt}
\setlength{\parskip}{2.8ex plus0.8ex minus0.8ex}

\begin{document}
	
\title{世界上所有的夜晚}
\date{}

\Large
\maketitle
%\vspace{-3cm}

\section{第一章  魔術師與跛足驢}

我想把臉塗上厚厚的泥巴,不讓人看到我的哀傷。

我的丈夫是個魔術師,兩個多月前的一個深夜,他從逍遙裏夜總會表演歸來,途經芳洲苑路口時,被一輛闖紅燈的摩托車撞倒在燈火闌珊的大街上。肇事者是個郊縣的農民,那天因爲菜攤生意好,就約了一個修鞋的,一個賣豆腐的,到小酒館喝酒划拳去了。他們要了一碟鹽水煮毛豆,三隻醬豬蹄,一盤辣子炒腰花,一大盤烤毛蛋,當然,還有兩斤燒酒。吃喝完畢,已是月上中天的時分了,修鞋的晃晃悠悠回他租住的小屋,賣豆腐的找炸油條的相好去了,只有這個菜農,惦着老婆,騎上他那輛破爛不堪的摩托車,趕着夜路。

這些細節,都是肇事後進了看守所的農民對我講的。他說那天不怪酒,而是一泡尿惹的禍。吃喝完畢,他想撒尿,可是那樣寒酸的小酒館是沒有洗手間的,出來後想去公廁,一想要穿過兩條馬路,且那公廁的燈在夜晚時十有八九是瞎的,他怕黑咕隆咚地一腳跌進糞坑,便想找個旮旯方便算了。菜農朝酒館背後的僻靜處走去。誰知僻靜處不僻靜,一男一女嘖嘖有聲地摟抱在一起親吻,他只好折回身上了摩托車,想着白天時走四十分鐘的路,晚上車少人稀,二十多分鐘也就到了,就憋着尿上路了。尿的催促和夜色的掩護,使他騎得飛快,早已把路口的紅燈當做被撇出自家園田的爛蘿蔔,想都不去想了,災難就是在這時如七月飛雪一樣,讓他在瞬間由溫暖墜入徹骨的寒冷。

街上要是不安紅綠燈就好了,人就會瞅着路走,你男人會望到我,他就會等我過去了再過。菜農說這話的時候,嘴角帶着苦笑。

小酒館要是不送那壺免費的茶就好了,那茶盡他媽是梗子,可是不喝呢又覺得虧得慌。賣豆腐的不愛喝水,修鞋的只喝了半杯,那多半壺水都讓我飲了!菜農說,哪知道茶裏藏着鬼呢!

菜農沒說,肇事之後,他尿溼了褲子,並且委屈地跪在地上拍着我丈夫的胸脯哭嚎着說,我這破摩托跟個瘸腿老驢一樣,你難道是豆腐做的?老天啊!

這是一位下了夜班的印染廠的工人、一個目擊者對我講的。所以第一個哭我丈夫的並不是我,而是“瘸腿老驢”的主人。

我去看這個菜農,其實只是想知道我丈夫在最後一刻是怎樣的情形。他是在瞬間就停止了呼吸,還是呻吟了一會兒?如果他不是立刻就死了的,彌留之際他說了什麼沒有?

當我這樣問那個菜農的時候,他喋喋不休地跟我講的卻是小酒館的茶水、燒酒、沒讓他尋成方便的那對擁吻的男女、紅綠燈以及那輛破摩托。這些全成了他抱怨的對象。他責備自己不是個花心男人,如果乘着酒興找個便宜女人,去小旅館的地下室開個房間,就會躲過災難了。他告訴我,自從出事後,他一看到紅色,眼睛就疼,就跟一頭被激怒的公牛一樣,老想撞上去。

我那天穿着黑色的喪服,所以他看待我的目光是平靜的。他告訴我,他奔向我丈夫時,他還能哼哼幾聲,等到急救車來了,他一聲都不能哼了。

他其實沒遭罪就上天享福去了,菜農說,哪像我,被圈在這樣一個鬼地方!

我看你還年輕,模樣又不差,再找一個算了!這是我離開看守所時,菜農對我說的最後一句話。他那口吻很像一個農民在牲扣交易市場選母馬,看中了一匹牙口好的,可這匹被人給提前預定了,他就奔向另一匹牙口也不錯的馬,叫着,它也行啊!

可我不是母馬。

我從來不叫丈夫的名字,我就叫他魔術師,他可不就是魔術師麼!十幾年前,我還在一所小學教語文,有一年六一兒童節,我帶着孩子們去劇場看演出。第一個出場的就是魔術師,他又高又瘦,穿一套黑色燕尾服,戴着寬檐的上翹的黑禮帽,白手套,拄一根金色的柺杖,在大家的笑聲中上場了。他一登臺,就博得一陣掌聲,他鞠了一個躬,柺杖突然掉在地上,等到他撿起它時,金色的柺杖已經成了翠綠色的了,他詫異地舉着它左看右看時,柺杖又一次“失手”落在地上,等他又一次撿起時,它變爲紅色的了。讓人覺得舞臺是個大染缸,什麼東西落在上面,都會改變顏色。誰都明白魔術師手中的物件暗藏機關,但是身臨其境時,你只覺得那根手杖真的是根魔杖,蘊藏着無限風雲。

我大約就是在那一時刻愛上魔術師的,能讓孩子們綻開笑容的身影,在我眼中就是奇蹟。

奇蹟是七年前降臨的。

由於我寫的幾篇關於兒童心理學方面的論文在國家級學刊上發表了,市婦女兒童研究所把我調過去,當助理研究員。剛去的時候我雄心勃勃地以爲自己會幹一番大事業,可是研究所的氣氛很快讓我產生了厭倦情緒。這個單位一共二十個人,只有四名男的。太多的做學問的女人聚集在一起絕不是什麼好事情,大家互相客氣又互相防範,那裏雖然沒有爭吵,可也沒有笑聲,讓人覺得一腳踩進了陰冷陳腐的墓穴。由於經費短缺,所有的課題研究幾乎很難開展和深入,我開始後悔離開了學校,我懷念孩子們那一張張葵花似的笑臉。研究所訂閱了市晨報和晚報,報紙一來,人們就像一羣飢餓的狗望見了骨頭,爭相傳閱。我就是在瀏覽晚報的文體新聞時,看到一篇關於魔術師的訪問,知道他的生活發生了變故的。原來他妻子一年前病故了,他和妻子感情深厚,整整一年,他沒有參加任何演出。現在,他準備重返舞臺了。我還記得在採訪結束時,魔術師對記者所講的那句話:生活不能沒有魔術。

我開始留意魔術師的演出,無論是在大劇院還是小劇場的演出,我都場場不落。我樂此不疲地看他怎樣從拳頭中抽出一方手帕,而這手帕倏忽間就變爲一隻撲棱棱飛起的白鴿;看他如何把一根繩子剪斷,在他雙手抖動的瞬間,這繩子又神奇地連接到了一起。我像個孩子一樣看得津津有味,發出笑聲。魔術師那張瘦削的臉已經深深地雕刻在我心間,不可磨滅。

有一天演出結束,當觀衆漸漸散去,他終於向臺下的我走來。他顯然注意到了我常來看他的表演,而且總是買最貴的票坐在首排。他對我說的第一句話是,你想學魔術?

我沒有學成魔術,我做了魔術師的妻子。

我們結婚的時候,他所在的劇團的演出已經江河日下,進劇場的人越來越少了。魔術師開始頻繁隨劇團去農村演出。最近幾年,他又迫不得已到一些夜總會去。那些看厭了豔舞、唱膩了卡拉OK情歌的男人們,喜歡在夜晚與小姐們廝混得透出乏味時,看一段魔術。有時看到興頭上,他們就把鈔票揚到他的臉上,吆喝他把鈔票變成金磚,變成女人的繡花胸衣。所以魔術師這幾年的面容越來越清癯,神情越來越憂鬱。他多次跟劇團的領導商量,他不想去夜總會了,領導總是帶着企求的口吻說,你是個男人,沒有性騷擾的問題,他們看魔術,無非就是尋個樂子,你又不傷筋動骨的;唱歌的那些女的,有時在接受獻花時還得遭受客人的“揩油”呢,人家順手在胸脯和屁股上摸一把,她們也得受着。爲了劇團的生存,你就把清高當成破鞋,給撇了吧!

魔術師只得忍着。他在夜總會的演出,都是劇團聯繫的。演出報酬是四六開,他得的是“四”,劇團是“六”。他常用得來的“四”,爲我買一束白百合花,一串炸豆腐乾或者是一瓶紅酒。

月亮很好的夜晚,我和魔術師是不拉窗簾的,讓月光溫柔地在房間點起無數的小蠟燭。偶爾從夢中醒來,看着月光下他那張輪廓分明的臉龐,我會有一種特別的感動。我喜歡他凸起的眉骨,那時會情不自禁撫摩他的眉骨,感覺就像觸摸着家裏的牆壁一樣,親切而踏實。

可這樣的日子卻像動人的風笛聲飄散在山谷一樣,當我追憶它時,聽到的只是瀰漫着的蒼涼的風聲。

魔術師被推進火化爐的那一瞬間,我讓推着他屍體的人停一下,他們以爲我要最後再看他一眼,就主動從那輛冰涼的跟擔架一樣的運屍車旁閃開。我用手撫摸了一下他的眉骨,對他說,你走了,以後還會有誰陪我躺在牀上看月亮呢!你不是魔術師麼,求求你別離開我,把自己變活了吧!

迎接我的,不是他復活的氣息,而是送葬者像漲潮的海水一樣涌起的哭聲。

奇蹟沒有出現,一頭瘸腿老驢,馱走了我的魔術師。

我覺得分外委屈,感覺自己無意間偷了一件對我而言是人世間最珍貴的禮物,如今它又物歸原主了。

我決定去三山湖旅行。

三山湖有著名的火山噴發後形成的溫泉,有一座溫泉叫“紅泥泉”,據說淤積在湖底的紅泥可以治療很多疾病,所以泡在紅泥泉邊的人,臉上身上都塗着泥巴,如一尊尊泥塑。當初我和魔術師在電視中看到有關三山湖的專題片時,就曾說要找某一個夏季的空閒時光,來這裏度假。那時我還跟他開玩笑,說是湖畔坐滿了塗了泥巴的人,他肯定會把老婆認錯了。魔術師溫情地說,只要人的眼睛不塗上泥巴,我就會認出你來,你的眼睛實在太清澈了。我曾爲他的話感動得溼了眼睛。

如今獨自去三山湖,我只想把臉塗上厚厚的泥巴,不讓人看到我的哀傷。我還想在三山湖附近的村鎮走一走,做一些民俗學的調查,收集民歌和鬼故事。如果能見到巫師就更好了。我希望自己能在民歌聲中燃起生存的火焰,希望在鬼故事中找到已逝人靈魂的居所。當然,如果有一個巫師真的會施招魂術,我願意與魔術師的靈魂相遇一刻——哪怕只是閃電的剎那間。

\pagebreak

\section{第二章 蔣百嫂鬧酒館}

我在烏塘下車了。不是我不想去三山湖,而是前方突降暴雨,一段山體滑坡,掩埋了近五百米長的路基,火車不得不就近停靠在烏塘。鐵路部門說,搶修最快要兩天時間。旅客們怨氣沖天,一會兒找車長要求賠償,一會兒又罵滑坡的山體是老妓女,人家路基並沒想摟抱你,你往它身上撲什麼呀。沒人下車,好像這列車是救生艇,下了就沒了安全保障似的。

在旅行中不能如期到達目的地,在我已不是第一次了,這裏既有不可抗拒的天氣因素,也有人爲的因素。有一次去綠田,長途客車就在一個叫黑水堡的寨子停了整整十個小時。茶農因不滿茶園被當地的高爾夫球場項目所徵用,聚集在交通要道上,阻斷交通,要向當地政府討一個“說法”。茶農們席地而坐的樣子,簡直就是一幅鄉野的夜宴圖。他們有的吃着涼糕,有的就着花生米喝燒酒,有的啃着蘿蔔,還有的嚼着甘蔗。最後政府部門不得不出面,先口頭答應他們的請求,他們這才離開公路。記得當地的交警呵斥他們撤離公路,說他們這樣做是違法的時候,茶農理直氣壯地說,霸佔了我們茶園就不算違法了?領導先違法,我們後違法,要是抓人,也得先抓他們!

烏塘是煤炭的產地,煤窯很多,空氣污濁。滯留在列車上的旅客開始向服務員大喊大叫,他們要免費的晚餐,那已是黃昏時分了。車窗外已經聚集了一些招攬生意的烏塘婦女,她們個個穿着質差價廉的豔俗的衣裳,不是花衣紅裙粉鞋子,就是紫衣黃褲配着五彩的塑料項鍊,看上去像是一羣火雞。她們殷勤地召喚列車上的人下車,都說自己的旅店的牀又幹淨又舒服,一日三餐有稀有幹、葷素搭配,有幾個男人禁不住熱湯熱水和牀的誘惑,率先下車了。我正在猶豫着,鄰座的一位奶孩子的婦女撇着嘴對她身旁的一個呆頭呆腦的男人說,這火車也真不會找地方壞,壞在烏塘這個爛地方!人家說這裏下煤窯的男人死得多,烏塘的寡婦最多。還真是啊,瞧瞧站臺上那些個女的,一個個八輩子沒見過男人的樣子!她鄙夷地掃了一眼那些女人,然後垂頭把奶頭從孩子的嘴裏拔出來,怨氣沖沖地說,我這對奶子攤上你們爺倆兒算是倒黴,白天奶小的,黑天喂大的,沒個閒着的時候!今晚有沒有飯還兩說着呢,小東西可不能把我給抽乾了!她懷中的嬰兒因爲丟了奶頭,哇哇哭鬧着。婦女沒辦法,只得又把那顆黑莓似的奶頭摁回嬰兒的嘴裏。嬰兒立刻就止了哭聲,咂着奶。女人罵,小東西長大了肯定不是個好東西,一個有奶就是孃的主兒!

烏塘寡婦多,而我也是寡婦了,婦女的話讓我做了下車的決定。我將茶桌上的水杯收進旅行箱,走下火車。

腳剛一落到站臺的水泥青磚上,就感覺黃昏像一條金色的皮鞭,狠狠地抽了我一下。在列車上,因爲有車體的掩護,夕照從小小的窗口漫進車廂,已被削弱了很多的光芒,所以感受不到它的強度。可一來到空曠之地,夕陽涌流而來,那麼的強烈,那麼的有韌性。光與光密集的聚合與糾集,就有了一股鞭打人的力量。

七八條女人的胳膊上來撕扯我,企圖把我拉到她們的店裏去。我選中了獨自站在油漆斑駁的欄杆前袖着手的一個婦女。她與其他女人一樣打扮得很花哨,一條綠地紫花的褲子,一件粉地黃花的短袖上衣。她的頭髮燙過,由於侍弄得不好,亂蓬蓬的,上面落了一層棉花絨子,看來她先前在家做棉活來着。她臉龐黑紅,皮膚粗糙,厚眼皮,塌鼻子,兩隻眼睛的間距較常人寬一些,嘴脣紅潤。她的那種紅潤不刺目,一看就不是脣膏的作用,而是從體內散發出的天然色澤。我撥開衆人朝她走去的時候,她衝我笑笑,說,你願意住我家的店麼?我說是。她上下左右地仔細打量了我一番,說,我家的店不高級,不過乾淨。我說這就足夠了。婦女又說,我沒有發票開給你。我說我不需要。她這才接過我的旅行箱,引領我走出站臺。

烏塘的站前廣場是我見過的世界上交通工具最複雜的了。它既有發向下轄鄉鎮的長途客車,還有清一色的夏利牌出租車,以及農用三輪車和腳踏人力車。最出乎意料的,幾掛馬車和驢車也堂而皇之地停泊在那裏。不同的是機械車排出的是尾氣,而馬車驢車排出的則是糞球。

婦女擤了一把鼻涕,把我領向西北角的一輛驢車。車上坐着一個仰頭望天的瘦小男孩,也就八九歲左右的光景。婦女吆喝一聲,三生,有客人了,咱回去吧!那個叫三生的男孩就低下頭來,怯生生地看着我。他穿一條膝蓋露肉的皺巴巴的藍布褲子,一件黃白條相間的背心,青黃的臉頰,矮矮的鼻樑,一雙豆莢似的細長眼睛透着某種與他年齡不相稱的憂鬱。婦女把箱子放在驢車上,把一張疊起的白氈子展開,喚我坐上去,而三生則拍了一下驢的屁股,說,草包,走了!看來“草包”是驢的名字。

草包拉着三個人和一隻旅行箱,朝城西緩緩走去。我問婦女要走多久。她說驢要是偷懶的話,得走二十分鐘;要是它順心意,十分八分也就到了。看草包那不慌不忙的樣子,我知道十分八分抵達的可能性是不存在了。不過,草包倒不像頭要偷懶的驢,它並不東張西望,只是步態有些踉蹌。它不是年紀大了,就是在此之前幹了其他的活兒而累着了。在一個陌生的地方,我喜歡這種慢條斯理的前行節奏,這樣我能夠更細緻地打量它的風貌。所以我覺得雄鷹對一座小鎮的瞭解肯定不如一隻螞蟻,雄鷹展翅高飛掠過小鎮,看到的不過是一個輪廓;而一隻螞蟻在它千萬次的爬行中,卻把一座小鎮瞭解得細緻入微,它能知道斜陽何時照耀青灰的水泥石牆,知道橋下的流水在什麼時令會有飄零的落葉,知道哪種花愛招哪一類蝴蝶,知道哪個男人喜歡喝酒,哪個女人又喜歡歌唱。我羨慕螞蟻。當人類的腳沒有加害於它時,它就是一個逍遙神。而我想做這樣一隻螞蟻。

烏塘的色調是灰黃色的。所有樓房的外牆都漆成土黃色,而平房則是灰色的。夕陽在這土黃色與灰色之間爬上爬下的,讓灰色變得溫暖,使土黃色顯得亮麗。街巷中沒有大樹,看來這一帶人注意綠化是近些年的事情,所以那樹一律矮矮瘦瘦的,與富有滄桑感的房屋形成了鮮明對照。正值下班高峯,街上行人很多。有的婦女挎着一籃青菜急急地趕路,而有的老頭則一手牽着放學的孩子,一手擎着半導體慢吞吞地走着。一家錄像廳張貼的海報是一對男女激情擁吻的畫面,從音像店傳出流行歌曲的節拍。酒館的幌子高高挑起,髮廊門前的臺階上站着叉着腰的招攬生意的染着黃頭髮的女孩子。這情景與大城市的生活相差無二,不同的是它被微縮了,質地也就更粗糲些、強悍些。所以有家旅館的招牌上公然寫着“有小姐陪,價格面議”的字樣,不似大城市的賓館,上門服務是靠入住房間的電話聯絡,交易進行得靜悄悄的。

草包穿城而過,漸漸地車少人稀,斜陽也凋零了,收回了纖細的觸角。腕上的手錶已丟失了二十分鐘,驢車卻依然有板有眼地走着。我知道婦女撒了謊,驢無論如何地疾走,十分八分抵達也是天方夜譚。婦女見我不驚不詫,倒不好意思了。她說,草包起大早拉了兩小時的磨,累着了,走得實在是太慢了。我便問她驢拉磨是做豆腐還是攤煎餅。婦女說做豆腐呀!接着她告訴我住她家的基本是熟客,老客人喜歡聞豆子的氣味。我明白她家既開豆腐房又開旅店,便稱讚她生意做得大。婦女說,大什麼大呀,不過一座小房子,前面當旅店,後面做豆腐房,賺個吃喝錢唄!我指着男孩問婦女,這是你兒子?婦女說,他是蔣百嫂的兒子,我家和他家是鄰居。我兒子可比他大多了,我十八歲就偷着結婚了,我兒子都在瀋陽讀大學了!她說這話時,帶着一種自得的語氣,我的心爲之一沉。我和魔術師沒有孩子,如果有,也許會從孩子身上尋到他的影子。就像一棵樹被砍斷了,你能從它根部重新生出的枝葉中,尋覓到老樹的風骨。

驢車終於停在一條灰黃的土路上,天色已經暗淡了。那是一座矮矮的青磚房,門前有個極小的庭院,栽種着一些雜亂無章的花草。路畔豎着一塊界碑似的牌匾,藍地紅字,寫着“豆腐旅店”四個字。婦女讓男孩卸下驢,飲它些水,而她則提着旅行箱,引我進屋。

這屋子陰涼陰涼的,想必是老房子吧。空氣中確實洋溢着一股濃濃的豆香氣,房間比我想像的要好,雖然七八平米的空間小了些,但牀鋪整潔,窗前還有一桌一椅。牀下放着拖鞋和痰盂,由於沒有盥洗室,門後放置着臉盆架。牆壁雪白雪白的,除了一個月份牌,沒有其他的裝飾,簡潔而樸素。窗簾也不是常見的粉色或綠色,而是紫羅蘭色的。沒有想到這個女人在打扮屋子上比打扮自己有眼力。

婦女說,這是單間,一天三十塊錢,廁所在街對面,晚上小解就用痰盂。飯可以在這裏吃,也可以到街上的小飯館。附近有五六個飯館,各有各的風味。她向我推薦一個叫暖腸的酒館,說是這家的魚頭豆腐燒得好。我答應着。她和顏悅色地爲我打來一盆洗臉水。簡單地梳洗了一番,我就出門去尋暖腸酒館了。

天色越來越暗淡,這座小城就像被潑了一杯隔夜茶,透出一種陳舊感。酒館的幌子都是紅色的,它們一律是一隻,要麼低低地掛在門楣上,要麼高高地掛在木杆上。一輛滿載煤炭的卡車灰頭土臉地駛過,接着一輛破爛不堪的麪包車像個乞丐一樣塵垢滿面地與我擦肩而過。跟着,一個推着架子車的老女人走了過來,車上裝着瓜果梨桃,看來是擺水果攤的小販。我向她打聽暖腸酒館,她反問我買不買水果。我說不買。她就一撇嘴說,那你自己去找吧。我便知趣地買了兩斤白皮梨,她這才告訴我,暖腸酒館就在前方二百米處,與雜貨店相挨着,不過“暖腸”的“腸”字如今被燕子窩佔了半邊,看上去成了“暖月”酒館。

當我提着梨尋暖腸酒館的時候,遇見了一條無精打采的狗。它瘦得皮包骨,像是一條流浪的狗。我摸出一隻梨撇給它,它吃力地用前爪捉住,嗅了嗅,將梨叼在嘴中,到路邊去了。它趴下來吃梨,而不是站着,看上去氣息懨懨的。

一對老人路過這裏,看見這狗,一齊嘆了口氣。老頭說,它這又是去汽礦站迎蔣百去了,主人不回來,它就不進家門!老太太則感慨地說,一年多了,它就這麼找啊找的,我看蔣百不回來,它也就熬幹油了。哪像蔣百嫂,這一年多,跟了這個又跟那個,聽說她前兩天又把張大勺領回家了!你說張大勺摞起來沒有三塊豆腐高,她也看得上!蔣百要是回來,還不得休了她!看來還是狗忠誠啊!

未見蔣百嫂,卻先見了她的兒子和她家的狗,這使我對蔣百嫂充滿了好奇。

暖腸酒館的“腸”字的右邊果然被燕子窩佔領了。窩裏有雛燕,燕媽媽正在餵它們。雛燕從窩裏探出光禿禿的腦袋,張着嘴等食兒。

未進酒館,先被一股炒尖椒的辣味嗆出了一個噴嚏,接着聽得一個女人大聲吆喝,再燙一壺酒來!我掀開門簾,進得門去。

酒館的店面不大,只有六張桌子,兩個大圓桌,四個小方桌。店裏只有三個酒客,兩男一女。兩個男人年歲都不小了,守着幾碟小菜對飲着。而坐在窗前方桌旁的女人則有好幾盤菜伺候着。見我進來,她揚起一條胳膊召喚我,說,姐們,過來陪我喝兩盅!她看上去三十來歲,穿一件黑色短袖衫,長臉,小眼睛,眼角上挑;厚嘴脣,梳着髮髻,胳膊渾圓渾圓的,看上去很健碩。她已喝得面頰潮紅,目光飄搖。我以爲碰到了酒瘋子,沒有理睬她,揀了一張乾淨的方桌坐下,這女人就被激怒了,她先是將酒盅摔在地上,然後又將一盤土豆絲拂下桌子。那地是青石磚的,它天生就是瓷器的招魂牌,酒盅和盤子立刻魂飛魄散。這時店主聞聲出來說,蔣百嫂,你又鬧了;你再鬧,以後我就不讓你來店裏吃酒了!蔣百嫂咯咯笑了,她用手指彈了一下桌子,說,我要是陪你睡一夜,你就不這麼說話了!店主看上去是個忠厚的人,他訕笑着搖頭,說,公安局這幫人也真是飯桶,你家蔣百丟了一年多了,活不見人,死不見屍,他們至今也沒個交代!蔣百嫂本來已經安靜了,店主的話使她的手又不安分了,她乾脆站了起來,掄起坐過的椅子,哐嚓哐嚓地朝桌上的菜餚砸去。辣子雞丁和花生米四處飛濺,細頸長腰的白瓷酒壺也一命嗚呼了。蔣百嫂邊砸邊說,我損了東西我賠,賠得起!那兩位酒客側過身子望了望蔣百嫂,一個低聲說,可惜了那桌菜;另一個則嘆息着說,女人沒了男人就是不行!他們並不勸阻她,接着吃喝了,看來習以爲常了。

蔣百嫂發泄夠了,拉過一把乾淨的椅子,氣喘吁吁地坐上去,像是剛逃離了一羣惡狗的圍攻,看上去驚魂未定的。店主拿着笤帚和撮子收拾殘局,蔣百嫂則把目光放到了窗外。暮色濃重,有燈火縈繞的屋裏與屋外已是兩個世界了。蔣百嫂忽然很淒涼地自語着,天又黑了,這世上的夜晚啊!

\pagebreak

\section{第三章 說鬼的集市}

旅店的女主人讓我叫她週二嫂,因爲她男人叫週二。我們研究所的蕭一姝,是個女權主義者。她在一篇文章中說,中國婦女地位的低下,從稱呼中就可以看出端倪。女人結婚生子後,雖然還有着自己的老名字,但是那名字逐漸被世俗的泥沙和強大的男權力量給淘洗乾淨了。她們雖然最終沒有隨丈夫姓,但稱謂已發生了變化,體現出依附和屈服於男權的意味,她認爲這是一種愚昧,是女性的一種恥辱。蕭一姝原來叫蕭玉姝,只因她丈夫的名字中也有一個“玉”字,便更名爲“蕭一姝”,她說女人接受由自己丈夫的姓氏得來的名字,就是一種奴性的體現。可我願意做相愛人的奴隸。可惜沒誰把我的名字依附在魔術師的名字上。

週二原先是礦工,一次瓦斯爆炸,他成了七人中惟一的倖存者,面部被嚴重燒傷,落了一臉的疤瘌。死裏逃生的週二再也不肯下井,用工傷賠償金和老婆開了豆腐店和旅店。週二做豆腐,挑到集市去賣,週二嫂則開旅店。週二每天凌晨三四點鐘就要起來趕着驢拉磨,做上幾板豆腐。週二賣豆腐,一賣就是一天。即使中午前他的豆腐擔子空了,他也不回家,仍混在集市中。跟掌鞋的聊家常啦,和修自行車的忙裏偷閒地下盤象棋了等等。週二嫂聽說我要蒐集鬼故事,就對我說,你不用挨門挨戶地尋,你跟着我家週二去集市,一天可以聽上好幾個鬼故事,那些出攤的小販子最喜歡講鬼故事了。週二眨巴着眼對週二嫂說,邢老婆子要在就好了,她說鬼說得好,可惜她也成了鬼了!史三婆也愛說鬼,不過比起邢老婆子那可差遠了,不過是《聊齋》中狐仙鬼怪的翻版!

我跟着週二去集市了。

週二個子不高,雖然他有力氣,但挑着一擔豆腐還是晃晃悠悠的。我跟在他身後,不斷地聽見別人跟他打招呼,週二,賣豆腐去啊?週二總是回一句,賣豆腐去!也有人跟他開玩笑,說,週二你行啊,白天吃自己的豆腐,晚上吃老婆的豆腐,有福氣啊!週二就啐一口痰,理直氣壯地說,我白天黑天吃的都是自家的豆腐,又不犯法,你說三道四個啥?!

太陽已經出來了,但它看上去面目混沌,裹在烏突突的雲彩中,好像一隻剛剝好的金黃的橙子落入了灰堆中。空氣中懸浮着煤塵,嗆得人直咳嗽。週二對我說,烏塘一年之中極少有幾天能看見藍天白雲,天空就像一件永遠洗不乾淨的衣裳晾曬在那裏。烏塘人沒人敢穿白襯衫,而且,很多人的氣管和肺子都不好。我問這附近有幾座煤礦?週二齜着牙說,大大小小總有二十幾個吧。我說政府不是加大力度清理小煤窯嗎?週二一撇嘴說,電視和報紙上是那麼說的,實際上呢,只要不出事,小煤窯是消滅不了的!開小煤窯的哪個不是頭頭腦腦的親朋好友?那等於給自己家設着個小金庫!礦工的命太賤了,前些年出事故死在井下的,礦長給個萬把的就把事兒給平了;現在呢,賠得多了些,也不過兩萬三萬的,比起命來,那算什麼!人死了,只要給了錢,沒人追究責任,照樣還有人下井,他們也照樣賺錢!

聽說週二在井下挖了六年煤,我便問他下井是什麼感覺?

週二說,啥感覺?每天早晨離開家,都要多看老婆孩子幾眼,下了井就等於踏進了鬼門關,誰能料到自己是不是有去無回?閻王爺想勾你的名字,大筆一揮,你就得留在地下了!媽的!

週二邊罵邊撂下擔子,一家小飯店的女主人吆喝住了他,要五塊豆腐。女主人顯然沒有睡足,頭髮沒梳理,趿拉着拖鞋,穿一件寬大的黃地藍花的棉布睡袍,呵欠連天的。週二麻利地將豆腐撮進女人遞過來的白鋁盆中。豆腐肌膚潤澤,它們“噗噗”地投入盆中,使盆底漫出一圈乳黃的水。女人忽然哈哈笑了起來,她對週二說,週二哥,你說蔣百嫂像不像這個盆子?它能裝土豆又能盛豆腐,能泡海帶也能擱蘿蔔絲,真是軟的硬的、黑的白的全不吝!我聽說她昨晚又鬧了酒館,把王葫蘆叫到家裏睡去了!你說王葫蘆都滿六十的人了,臉比驢還黑,天天撿破爛,一年到頭洗不上一回澡,跟他睡,不是睡在廁所裏又是什麼!

週二聽女人這樣議論蔣百嫂,有些惱了,他說,你也不要把自己說得那麼幹淨,你家劉爭一跑長途,朱鐵子不就老來你店裏吃酒麼,一吃就是一夜,誰不知道?!你們這些女人啊,就跟蚯蚓一樣,不能讓你們見天光,埋在土裏你們安分守己;一挖出來,就學會勾引人了!

蚯蚓勾引的是魚!那女人大聲地辯駁。她受了奚落倒也不惱,只是不再呵欠連天了。她對週二說,我知道你對蔣百嫂好,都說你是蔣三生的乾爹,一家人哪有不向着一家人的?!

週二挑起擔子,衝女人撇撇嘴,走了。跟着他走的,有被汽車挾起的塵土、陳舊的陽光和我。也許還有匍匐的螞蟻也跟着,只不過沒有被我們注意到罷了。

烏塘有三個集市,週二說我來的集市規模居中,另兩個集市,一個比它大,一個比它小。比它大的集市有服裝和日用小百貨賣,比它小的只賣些肉蛋禽類、蔬菜瓜果。

週二進了集市,就像一隻鳥進了森林,自由而快活。他和老熟人一一打招呼,將擔子卸在他的攤位上。已經有很多小商販出現在集市上了,賣糖酥餅和綠豆稀飯以及油條和豆漿的攤位前人頭攢動,生意紅火。怪不得我要在旅店吃早飯時,週二對週二嫂說,她不是要跟着我去集市聽鬼故事麼,還不如在那兒吃呢!想吃棗泥餅有棗泥餅,想喝豆腐腦有豆腐腦,想吃水煎包有水煎包!當時週二嫂白了週二一眼,說,你吃慣了集市的早飯,嫌棄我的手藝了!週二連忙賠着笑臉說,哪能呢,你做的飯我這輩子吃不夠,下輩子還想吃呢!週二嫂笑了,她擰了一把週二的臉,說,就你這一臉的疤瘌,也只能可着我的飯來吃了,別人誰得意你?他們滿懷愛意的鬥嘴使我想起魔術師,以往我們也常這樣甜蜜地鬥嘴,可那樣的話語如今就像鐫刻在碑上的墓誌銘一樣,成爲了永恆。

我到小食攤前吃了碗黑米粥和一個餡餅。有一個食客對着免費的鹹菜大嚼大嚥着,瘦削的攤主用眼睛白着他,說,不怕?着啊?食客說,?着就喝水!攤主說,水也得花錢啊。食客說,喝水便宜。攤主又說,喝多了水找公廁撒尿也得花錢啊。食客被激怒了,他把鹹菜罐摔在地上,罵,免費的鹹菜你不叫吃,乾脆收費得了,別死要面子硬撐着,還叫男人嗎?!攤主看着碎了的鹹菜罐,居然委屈得落淚了。他穿件藍背心,戴一條油漬斑斑的綠圍裙,黑紅的臉龐,看上去像是一隻被做成了醬菜的細長的青蘿蔔,顏色暗淡,散發着一股陳腐的氣息。他這一哭,食客倒了胃口,他放下筷子,將一張十元錢拍在桌子上,說,不用找了,就頭也不回地走了。與他相鄰的賣豆腐腦的說那攤主,你合適啊,這一頓早飯也就三塊兩塊的,你一傢伙得了十塊,頂三個人吃的了,昨晚一定夢見金鯉魚了吧?攤主抽搐着臉說,除了金秀,我還能夢見誰?賣豆腐腦的說,金秀又跑你的夢裏去了?我看你趕快再找一個算了,她沒了三年了,你天天睡涼炕,她當然記掛着你了!要是你娶了新的,她也就過她的陰日子去了,人家在那裏也可以再找一個,你不找,也耽誤人家啊!

聽他們這一番話,我知道這個面容悽苦的男人死了老婆,而且他與老婆感情深篤。我便膽怯地問他,死了的人進了活人的夢中,會是什麼樣子?魔術師在時,我倒時常夢見他;可他永別我後,我的腦子一片混沌,沒有什麼具體的影像,他把我的夢想也帶走了。

攤主淚眼朦朧地望了我一眼,嘴脣哆嗦了幾下,說,死了的人回到活人的夢中,當然是活着時的樣子了!她會囑咐你風大時別忘了關窗,下雪了別忘了給孩子戴上棉帽子。唉,她也真是命苦,死了還得跟我操心!

來了兩個身上掛滿了石灰點的民工,攤主擦乾眼淚,招呼他的生意去了。我回到週二那裏,他正在吸菸。我問那個攤主的老婆是怎麼死的?週二噴出一口青煙說,他老婆得了痢疾,就到家跟前的個體診所打點滴。你說青黴素這東西也真是邪性,點了不出兩小時,人就沒氣了!人家說,診所的老周沒有給她做過敏試驗,人才死了。我看這女人也是命薄,拉肚子本不是大毛病,拉不死人,非要去診所,這下好,因小失大,把命都搭上了!

診所的那個姓周的呢?我問。

他呀,原先是個獸醫,這些年得病的人比得病的牲畜要多,他就換下藍袍子,穿上白大褂,掛上聽診器,開起了診所!他也有點能耐,治好過一個偏頭疼的女人,還治好過幾個人的胃病,所以他沒出事時,生意還挺紅火的!

他一個當獸醫的,怎麼會拿到爲人看病的行醫執照呢?我問。

嗨,這世道的黑白你還看不清哇,有錢能使鬼推磨唄!週二吐了口唾沫,說,老周的連襟在衛生局當局長,拿個行醫執照,就跟從自家的樹上摘個果子一樣輕而易舉,有什麼難的?出了事後,人家花了兩萬塊,就把事平了!就說人不是點滴死的,是心臟病發作死的!

這男人也就同意了?我瞟了那攤主一眼。

不認又怎麼着?打官司他打得起嗎?反正他老婆已進了鬼門關,還不如弄倆錢,將來留着給孩子用!週二嘆了口氣,指着那攤主說,他原來是個挺樂和的人,老婆沒了,就變得跟女人一樣愛計較了,動不動還哭,哪還有點男人的樣子!

老周呢?我心灰意冷地問。

他呀,在這兒混不下去了,早就走了。聽說去了蕪湖的親戚家,不幹這行了,養蝦去了,誰知道呢?週二又嘆了一口氣,說,在這個集市上,辛酸的人海着去了,你要聽鬼故事,隨便逛逛就能聽到。

我與週二閒談的時候,已經有兩個人買了豆腐走了。但凡做小本生意的,都是些眼疾手快的人,他們能心、手、口並用,嘴上抽着香菸並且與你講着故事,手上麻利地打理着生意,什麼也不耽誤。

集市越來越熱鬧了。推着架子車、挑着貨擔的生意人越聚越多,先前還空着的攤牀也就沒有閒着的了。由於這集市有個長條形的頂棚,集市邊緣的攤牀點染着陽光,而中心地帶則相對暗淡些,陽光未爬到那裏就斷了氣。週二把我引向集市中央陰涼處的一個攤牀,對一位坐着的袖着手的穿黑衣的老女人說,史三婆,這是我家客人,想蒐集鬼故事,你給她講幾個吧!你知道那麼多的鬼故事,不講不就全爛肚子裏了麼?史三婆呸了週二一口,說,我的故事值錢,講一個得給我十元!週二說,明天我給你炸包豆腐泡吃,頂了講故事的錢了!史三婆上上下下地打量了我一番,說,你給哪裏蒐集鬼故事?我說爲自己。史三婆就打了一個嗝對我說,你又不是從陰間來的,蒐集那故事做啥?我想與她有個輕鬆的談話氛圍,就開玩笑說,誰說我不是從陰間來的?我這話沒嚇着史三婆,倒把與她相鄰的賣笤帚的女孩給嚇着了,她驚叫着說,史三婆,我一看她的樣子就像個鬼,一身的黑衣服,瘦得全是骨頭,臉上沒血色,你可別讓她靠近咱們呀!史三婆笑了,她從容不迫地說,鬼就是鬼,哪能讓你看得着呢!你不用怕。史三婆讓我到攤牀裏面去坐,不然我像根柱子似地戳在她面前,影響她的生意。我笑了笑,從通道旁的小便道走到攤牀裏面。也許是久已不笑了,我的笑不但使自己起了寒意,也讓那個女孩打了個哆嗦。史三婆的攤牀上,擺着形形色色的滅害劑,有毒鼠強、滅蠅水、驅蚊油、除蟑靈、敵殺死等等。史三婆的鬼故事,就以毒鼠強爲背景而開始了。

有個年輕的寡婦,她男人死於礦難的“冒頂”事件。她攤上個好吃懶做又心狠手毒的婆婆,一日伺候不周,婆婆就趁她熟睡時用針扎她的額頭。寡婦受夠了婆婆的氣,就買了兩包毒鼠強,燉了一鍋肉,打算與婆婆同歸於盡。那天下着大雨,電閃雷鳴的,寡婦早把孩子打發到姐姐家去了。她盛了肉,放在桌子上,又取了兩個酒杯和兩雙筷子,喚婆婆喝酒吃肉。婆婆那時正站在窗前把一杯陳茶往窗外潑,聽見兒媳喚她,她回身便罵,我知道你有貳心了,想今晚把我灌醉,好在我兒子睡過的炕上養漢!寡婦忍着,沒有和婆婆頂嘴,想引誘她把肉吃了。這時外面的雷聲越來越響,窗櫺被震得跟敲鑼似的,咣咣響,寡婦突然看見他丈夫從窗口飄了進來,就像一朵烏雲。她剛叫了一聲丈夫的名字,那朵雲就化做一道金色的閃電,像一條繩子一樣,勒住了她婆婆的脖子。婆婆倒地身亡,被雷電取走了性命。寡婦明白這是丈夫在幫助她,如果她也死了,孩子誰來管呢?從那以後,這寡婦就守着孩子過日子,沒有再嫁。而她的孩子也爭氣,幾年後考上了一所名牌大學。

史三婆的話使我聯想到魔術師,他也會化做一道閃電嗎?看來以後的雷雨天氣我得敞開窗口了,也許我的魔術師會挾着一束光焰來照亮我晦暗的眼睛。

賣笤帚的女孩發現我對鬼故事確實有着與人一樣的着迷,她不再懷疑我是鬼了,她接着史三婆,講了另一個鬼故事。

我表哥在烏塘自來水公司當司機,他有一個朋友叫賈固,在法院工作,是法警。有一年冬天,賈固的車掉進雪窩裏,喚我表哥幫他拖出來。我表哥和賈固怕耽誤上班,凌晨三點就上路了。那輛車陷在一片墳地裏,天落着雪,四周白茫茫的。表哥拖着拖着車,忽然見雪野中閃出一個人影,是個女人,她戴着白圍巾,白帽子,臉盤素淨,面容秀麗,說要搭我表哥的車進城。在那樣一個荒僻的地方,突然出現這麼一個女人,我表哥覺得蹊蹺,就問她怎麼這麼早就來到野外?那女人只是笑,並不出聲。再問她是人是鬼時,她擺擺手就消失了。表哥嚇得腿直哆嗦,他們把車拖出來,再也不敢回頭看一眼墳場。表哥跟賈固說,他當法警,一定是槍斃錯了人,冤魂纔會從墳地飄出來。賈固便把由他親手斃掉的死刑犯一一過篩子,最後真的找到了那個面容如墳地上出現的女人的照片,她在七年前就被處決了。存檔的卷宗說她紅杏出牆,殺害了丈夫。賈固認爲這案子判得肯定有不公之處,就暗中複查舊案。從此他寢食不安,衣冠不整,漸漸地精神不太正常了,常指着妻子叫老孃,指着饅頭叫靈芝。前年冬天,他被一輛運煤的卡車撞死了。表哥說在賈固的葬禮上,他又看見了那個在墳地遇見的女人,她還是那麼年輕,戴着白帽子,白圍巾,一言不發。表哥想跟她說幾句話,可她一轉眼就在賈固的靈前消失了。直到今年春天,派出所抓到了一個盜竊犯,他交代出自己幾年前因搶劫未果,殺了一個人,而那個人就是那個女人的丈夫。看來她確實是被屈打成招,含冤而死的。賈固殺了本不該被殺的人,她也就取走了他的性命。你說以後誰還敢當法警啊?

女孩講故事的能力十分了得,而這個鬼故事則讓我起了寒意。我誇讚她口才好,史三婆咳嗽了一聲,說,她考上了大學,口才自然差不了!我便問她既然考上了大學,爲什麼不去上?女孩別過臉去,臉上現出淒涼的神色。史三婆說,還不是因爲窮?她媽是個藥簍子,他爸呢,常年下礦井,落了一身的病,如今風溼病重得連路都走不了,只能躺在炕上。一家兩個病號,哪有錢供她上學呢?

那爲什麼不向社會尋求救助呢?我問。

像她這樣上不起大學的孩子又不是一個,救助得過來麼?史三婆說,這丫頭出來做小買賣,說掙了錢供自己上大學。我看靠她賣笤帚,賣到人老珠黃了也上不起!還不如學那些來烏塘“嫁死”的女人,熬它個三年五載的,“嘭——”地一聲,礦井一爆炸,男人一死,錢也就像流水一樣嘩嘩來了!要說什麼是鬼,這纔是鬼呢!史三婆氣咻咻地拈起一瓶滅蚊劑,漫無目的地噴了一下,好像我是隻吸人血的毒蚊似的。

女孩淚眼朦朧地對史三婆說,我纔不“嫁死”呢!

我問,什麼叫“嫁死”?

史三婆擤了把鼻涕,突然指着從不遠處走來的一個染着棕紅頭髮的穿花衣的女人說,這媳婦就是來烏塘“嫁死”的。可她嫁來三年了,她男人還活靈活現着!聽人說她一個白天都在外面打麻將,晚上回家一看到她男人從井下平安回來了,她就嘆氣,連飯也不做給他吃。

我大惑不解,問,這是爲什麼?

史三婆鄙夷地看着那個走得愈來愈近的女人,說,你是外地人,當然就不知道“嫁死”是怎麼回事了。烏塘不是礦井多,事故多麼,這些年下井死了的礦工,家屬得到的賠償金多,一些窮地方的女人覺得這是發財的好門路,就跑到烏塘來,嫁給那些礦工。他們給自家男人買上好幾份保險,不爲他們生養孩子,單等着他們死。我們私下裏就管這樣的女人叫“嫁死的”。前年井下出事故時,你看吧,那些與丈夫真心實意過日子的女人哭得死去活來的,而外鄉來的那些“嫁死的”呢,她們也哭幾嗓子,可那是乾嚎,眼裏沒有淚,這樣的女人真是鬼呀!

那個遭史三婆貶損的女人走到攤牀前了,她拿起一瓶敵殺死,問,多少錢?史三婆說九塊。那女人嘟囔道,不是六塊麼?史三婆抿了一下額前的頭髮,說,賣給你就是九塊,愛買不買!女人撇下瓶子,說,又不是你一家賣敵殺死!她瞪了史三婆一眼,離開了攤牀。我望着她的背影,看着她嫋娜的腰肢和裸露着的性感的胳膊,有一種分外寒冷的感覺。

史三婆的生意在九點以後開始興旺了。看來烏塘夏季的蚊蠅很多。買滅害藥的百分之九十都是女人。史三婆沒忘了見縫插針地給我講故事,什麼女人死後變成了狐狸,迷死了獵人;什麼大姑娘睡在花樹下,無緣無故地懷上了鬼胎,這孩子出生後是個混世魔王,無惡不作。可我對這些傳說的鬼故事已經不感興趣了。集市上人影憧憧,誰能想到有一些卻是鬼影呢?!炸油糕與麻花的甜香氣,與炸臭豆腐乾的氣息混合在一起;賣瓜果蔬菜的與賣糧油副食的爭先恐後地吆喝着,地面漸漸地積了瓜子皮、紙屑、菸蒂、菜葉等遺棄物,當然還有人們隨口吐出的痰。

蔣百嫂也出現在集市上了。史三婆告訴我,她男人蔣百失蹤後,她就來集市賣油茶麪兒了。她是集市中來得最晚的生意人,因爲她夜晚老是喝酒後帶男人回家鬼混,所以起得遲。她說蔣百嫂的油茶麪生意還不錯,男人們很喜歡猴在她的攤牀前。蔣百嫂仍是一襲黑衣,綰着髮髻,嘴裏嚼着什麼,胳膊上挎着一個木桶,木桶裏裝着油茶麪。她看人時的目光是迷茫的、懶散的,步態微微踉蹌,似乎還沒醒酒的樣子。她穿行在集市中,就像一股凜冽的風掠過湖面,泛起寒波點點,很多人都擡着眼望她,就像看戲中人似的。

\pagebreak

\section{第四章 失傳的民歌}

烏塘的雨是我見過的世界上最骯髒的雨了,可稱爲“黑雨”。雨由天庭灑向大地的時候,裹挾了懸浮於半空的煤塵,雨便改變了清純的本色。烏塘人因而喜歡打黑傘。衆多的打黑傘的人行走在縱橫交錯的街巷中,讓人以爲烏塘落了一羣龐大的烏鴉。即便如此,雨過天晴,烏塘還是顯得清亮了許多。

週二聽說我想蒐集民歌,就讓我到回陽巷的深井畫店去。他說畫店的主人陳紹純,最喜歡唱民歌了。不過他唱的歌有點悲,人們都說那是“喪曲”。他老婆不允許他在家唱,他就在畫店唱。回陽巷的商販,最不喜歡與他爲鄰了。你這邊生意剛開張,那邊就傳來了他唱喪曲的聲音,誰不忌諱呢。所以毗鄰畫店的商鋪,從燒餅鋪到狗肉店再到理髮店,已經幾易其主。如今與它相挨的,是家壽衣店。

週二嫂套上驢車,和蔣三生到火車站招攬生意去了。三生騎在家裏的屋頂上,週二嫂喊他的時候,他激靈了一下,差點一個跟頭從屋頂跌下來。週二嫂對我說,自從蔣百失蹤後,這孩子就不愛呆在屋裏,他除了喜歡到旅店玩,還愛坐在自家的屋頂望天。有的時候他在屋頂一坐就是一下午,似乎在張望他父親歸來。

蔣百是如何失蹤的呢?聽週二說,蔣百在小鷹嶺礦採煤,是個性情溫順的人。下礦歸來,他愛喝上幾盅酒,蔣百嫂因而練就了一手做下酒菜的好手藝。小鷹嶺是個大礦,一共有六個作業點,每個作業點都要有一到兩個班次在作業,而每班次是十人。礦井出事那天,蔣百早晨時離開家去礦上了,可他傍晚沒再回來。從蔣百所在的班次的事故工作面上找到了九具屍體,惟獨沒有蔣百的。礦長說,蔣百那天根本沒有到小鷹嶺,下井的是九個人。這麼說,蔣百那天是去別的地方了。他雖然倖免於難,但是形跡杳然,沒人知道他去哪兒了。大家對蔣百的失蹤有多種猜測,有人說他拋棄了蔣百嫂,尋他中學時的相好去了;有人說蔣百被人害了,行兇者早已將他焚屍滅跡。還有更荒唐的說法,說蔣百厭倦了井下生活,到深山古剎做和尚去了。蔣百嫂原先是個羞澀的人,蔣百失蹤後,她變了一個人似的,三天兩頭就去酒館買醉,花錢大手大腳的,人也變得浪蕩了,隔三差五就領男人回家去住。烏塘的許多女人因而敵視蔣百嫂,怕自家男人被她勾引了去。蔣百嫂原來受僱於一家託兒所,給人看小孩子,蔣百失蹤後,她就到集市賣油茶麪去了。

週二告訴我,派出所曾對蔣百失蹤的事,調查過一些人,問他們在礦難的那天是否見過蔣百?結果有兩個人見過他,一個是糧庫的退休工人老周頭,一個是郵局的顧小栓,他們都說蔣百那天早晨穿着藍色的工作服,戴着礦帽,去汽礦站搭乘礦車。蔣百身後,還跟着他家的狗。它每天早晨忠心耿耿地把蔣百送上礦車,黃昏時再跑到礦車停靠地,歡天喜地地把主人迎回來。所以蔣百失蹤後,這狗就不入家門,依然在傍晚時去接主人。礦車一停下,它就湊上前,但下車的人總是讓它失望。它以前威風凜凜的,如今卻憔悴不堪,烏塘人因而喜愛這條忠實於主人的狗,一些飯館的老闆見它從街巷中走來,常撇一些香腸和牛肉給它。

回陽巷是一條幽長的巷子,深井畫店就在這巷子的盡頭,果然與一家壽衣店相鄰着。畫店很小,有一扇西窗,西北角的棚頂打着一個菱形木方,木方下垂下來幾條鐵鏈,鉤着幾幅畫。我見過的畫店,畫都是懸掛在牆壁或者是倚在牆角的,沒有像深井畫店這樣把畫吊在棚頂下的,這做派倒有些像肉鋪和洗染店了。畫店的東北角,是個一丈見方的櫃檯,一個面容清癯的老人正俯在那兒畫着什麼。聽見門響,他皺了一下眉,但並未擡頭。我問他,您就是陳紹純先生嗎?他仍未擡頭,而是抽了一下嘴角,微微點了點頭。我湊到櫃檯前,見他正在畫荷。那荷花沒有一枝是盛開着的,它們都是半開不開的模樣,嬌弱而清瘦。我只能訕訕地自我介紹,說我想做點民俗學的調查,蒐集民歌,聽週二介紹他民歌唱得好,特來拜訪。我說話的時候,他始終沒有望我一眼,所以我覺得是隔着竹簾與他講話。見他態度如此傲慢,我正想走掉,他突然放下畫筆,沒容我有任何心理準備,他一歪脖子,歌聲就如倏忽而至的漫天大雪一樣飄揚而起。我頭一回聽人唱沒有歌詞的歌,它有的只是旋律。那歌聲聽起來是那麼的悲,那麼的寒冷,又那麼的純淨,太不像從大地升起的歌聲了。

他的歌聲起來得突然,走得也突然,當我還爲着歌聲的那種無法言說的美而陶醉時,它卻戛然而止了。他低聲問了句,這樣的悲調你也想收集麼?如今悲曲上不了檯面,你沒見電視中唱民歌的個個都是歡天喜地的?

我說,我喜歡這悲調。我的話音剛落,一個穿着肥大褲衩、着一件油漬漬藍背心的壯漢滿面流汗地推門而入。他胖得兩腮的肉直往下墜。他的腋下夾着一幅玻璃框風景山水畫。他一進來就嚷嚷,陳老爺,我娘嫌這牡丹不鮮豔,你再給上上色,多塗點紅啊粉啊的!

陳紹純擡起頭,對來人說,牛枕,你回去告訴你娘,牡丹塗紅塗得重了,那不成了猴子的屁股了嗎?我深井畫店就是這麼個畫法,她又不是不知道!她要是不稀罕,我將畫收回,錢一分不少還給她,你看行不行?

牛枕將畫擺在櫃檯上,撩起背心一角,揩臉上的汗。他粗聲大氣地說,哎喲,陳老爺,我娘就認你的畫,別人畫的她還不得意呢!她癱了三年了,整天看的是牆,我早就說要給牆掛上幾張畫讓她看,可她嫌礙眼、累贅,今年她是頭一回提出要看畫,點着名要看你畫的牡丹,她年歲大了,眼神哪比年輕人,常把貓看成老鼠,把人看成雞毛撣子。你畫的紅牡丹,她看成了粉的;粉的呢,又看成白的了!我又沒那兩把刷子,不然我就給牡丹上色了。陳老爺,求您了,改天我割一塊好肉來孝敬您!

陳紹純嘆了口氣,說,再上色,可不就是糟踐了那些牡丹麼!你留下畫吧,明天上午來取。

牛枕像小孩子一樣興高采烈地拍着手,說,謝謝陳老爺!我娘看的牡丹,就得是歌廳中那些坐檯的小姐,臉上得擦上二兩粉,頭髮抹上二兩油,嘴脣塗上二兩口紅,濃濃的,豔豔的,不然她是不看的!

陳紹純說,我看你在集市賣了兩年肉,嘴皮子也練出來了。

牛枕說,我不學會吆喝,賣的就是天鵝肉,也得爛在攤牀上,如今這世道,叫喚的鳥兒纔有食兒吃呢。

陳紹純對牛枕說,明天來取畫,順便爲他在集市買兩斤蔣百嫂賣的油茶麪。

一提蔣百嫂,牛枕就眉飛色舞地訴說剛剛發生在集市的一件事,蔣百嫂把一個小媳婦的門牙打掉了,這是個來烏塘“嫁死的”外鄉女人。那女人買油茶麪,蔣百嫂不賣給她,說她的油茶麪不能給黑心爛肺的人吃。小媳婦很厲害,她朝蔣百嫂身上吐了口唾沫,說烏塘有一個爛貨,她男人失蹤後,她熬不住了,連撿破爛的老頭都能和她睡上一覺,這個爛貨怎配指責別人?蔣百嫂便大打出手,咣咣幾拳,將“嫁死的”打得鼻青臉腫,口吐鮮血,掉了顆門牙。小媳婦哭嚎着,打電話報了警。派出所的民警趕到集市後,見是蔣百嫂在惹是生非,就說她,你看烏塘哪個女人像你?鬧了酒館又鬧集市,還有一點做女人的樣子麼?!蔣百嫂一生氣,就把一碗剛衝好的油茶麪潑到民警臉上,燙得民警跟挨宰的豬一樣嗷嗷叫。牛枕說完,哈哈笑了起來。

陳紹純說,蔣百嫂這回可闖了大禍了,那“嫁死的”小媳婦丟了顆門牙,還不得訛她個千兒八百的?

牛枕說,蔣百嫂有那麼多男人供着,賠她個萬把的也不在話下!再說了,派出所這幫吃閒飯的找不到蔣百,愧對蔣百嫂,也不敢把她怎麼着!

看來在烏塘,蔣百嫂因爲蔣百的失蹤而成了新聞人物,你走到任何角落,都能聽到她的消息。

牛枕走了,陳紹純依然畫他的荷花。他垂着頭,凝神貫注。也許在他眼中,我就是這畫店的靜物。我想也許他畫完荷花,就有與我談天的興致了。

我走出深井畫店時,覺得帶着一身的雪花,是陳紹純歌聲中的音符附着在我身上了。太陽在厚薄不一的雲中徘徊,遇到雲薄的地方,它就淺淺微笑着,而到了雲厚之處,它就像一個蒙面的修女,一臉的肅穆。大地也因此忽明忽暗着。我不知道我的魔術師是否在雲層的後面,他仍如過去一樣在溫柔地注視着我麼?太陽與月亮之所以永遠光華滿面,是不是容納了太多太多往生者的目光?有一縷雲,輕飄疏朗得特別像一片鵝毛,它令我想起婚姻生活中那些美好的日子。每當假日時我垂着窗簾放縱地睡懶覺時,已經把早飯熱了不知幾遍的魔術師就會捏着一片雪白的鵝毛,輕輕地撩撥我的臉,把我叫醒。那片鵝毛是他變魔術的道具,他在舞臺上,能用它變出手帕和棒棒糖。我被擾醒後,總是捏着他的鼻子不許他喘氣,嗔怪他斷送了我的美夢。魔術師就會旋轉着鵝毛,大張着嘴吃力地對我說,你睡了一夜,睫毛都是眵目糊,我爲你掃一掃還不應該啊?他是把鵝毛當成了笤帚,而把我的睫毛當成了庭院前的柵欄了。他去世後,那片鵝毛被我插在他的指縫間,隨他一起火化了,因爲再也不會有其他男人用這片鵝毛叫我甦醒了。

我在異鄉的街頭流淚了。只要想起魔術師,心就開始作痛了。一個傷痛着的人置身一個陌生的環境是幸福的,因爲你不必在熟悉的人和風景面前故做堅強,你完全可以放縱地流淚。

我哭泣着,漫無目的地走着。一些行人發現我滿面淚痕的樣子,現出怪異的神色。有兩個人還關切地詢問我,一個問我是不是丟了東西。一個問我是不是得了絕症。我回答他們的不是話語,而是綿綿不絕的淚水。我邊走邊看天,直到那片鵝毛般的雲蕩然無存了,才注意看腳下的路。過了回陽巷,是紫雲街。我很喜歡烏塘街巷的名字,它沒有那麼大衆的名字,比如很多城市都有的“前進路、中山路、勝利街、光芒巷、衛東巷”等等,烏塘街巷的名字,很像一個坐在夕陽底下飽經風霜又不乏浪漫之氣的老學究給起的,如青泥街、落霞巷、月樹街等。除了紫雲街外,我還喜歡月樹街的名字。月樹街上有幾家歌廳,我踅進兩間,問這裏可有唱民歌的。經營者便問我,你想點民歌?他們盛情地從KTV包房中取出點歌本,向我推薦《山丹丹花開紅豔豔》《走西口》《小放牛》《十送紅軍》《蘭花花》《趕牲靈》等歌,我說我想聽那種沒有被流傳下來的民歌,他們就像打量怪物一樣對我說,那你走錯地方了。

我確實走錯地方了。雖然歌廳的營業高潮還未到來,但偶爾飄來的絲絲縷縷歌聲,都是那些濫俗怪誕的流行歌曲。流行歌曲有兩類最走紅,一種是聲嘶力竭地如排泄不暢地沙啞着嗓子吼,一種是嗲聲嗲氣地軟着舌頭跟蚊子一樣地哼哼。這樣的歌聲在我聽來就是人間的噪音。最後在一家名爲“星星”的歌廳,總算聽到一首三十年代的老歌《陋巷之春》,才讓我獲得了某種慰藉。唱它的是一個二十上下的女孩,雖然她模仿周璇的那種清純甜美有些誇張,但那旋律本身的美好卻像一條奔涌而來的清流一般,難以抵擋。我很喜歡它的歌詞:

人間有天堂,天堂在陋巷。春光無偏私,佈滿了溫暖網。樹上有小鳥,小鳥在歌唱。唱出讚美詩,讚美青春浩蕩。

鄰家有少女,當窗曬衣裳,喜氣上眉梢,不久要做新娘。春色在陋巷,春天的花朵處處香。我們要鼓掌,歡迎這好春光。

我坐下來,在光怪陸離的燈影下要了一杯奶茶,聽完了這首歌。之後,又回到月樹街。

月樹街上的行人多了,黃昏已近,人們都在歸家,街市比先前嘈雜了。我到一家麪館要了碗炸醬麪,吃過後又進了一家茶館,喝了杯綠茶。茶杯油漬漬的,讓人覺得店主是開肉食店的而不是開茶館的。等我再回到月樹街時,天色已昏,歌廳的霓虹燈開始閃爍了,流動的商販也出現了,他們賣的貨色品種繁雜,有賣燒餅和牛肉的,也有賣棉花糖、頭飾、背心短褲、果品以及二手手機和盜版書籍的。我買了一摞燒餅,一塊醬牛肉,又到一家超市買了一瓶二鍋頭,朝回陽巷走去。我還想在這樣的日落時分聆聽幾首民歌,再沾染一身雪花的清芬之氣。

快到畫店的時候,我見與它相鄰的壽衣店走出來兩個臂戴黑紗的人,他們擡出一隻大花圈。那些紫白紅黃的花朵被晚風吹得簌簌響,使我想起魔術師的葬禮。也有很多人送了花圈給他,可我知道他最不喜歡紙花了,我差人將他靈堂所有的花圈都清理出去。我知道有我爲他守靈就足夠了,我是他唯一的花朵,而他是這花朵唯一的觀賞者。

我推開畫店的門,見陳紹純正坐在西窗下打盹,櫃檯上空空蕩蕩的,看來他已畫完了荷花。店裏光線虛弱,可他沒有開燈。從他蹙眉的舉止中,可看出他知道有人進來了,可他並未擡頭,仍舊眯着眼。我輕輕走過去,將酒菜擺在他腳畔,說,該吃晚飯了。

他睜開眼,微微擡了擡頭,看了看我,又看了看酒菜,嘆了一口氣,說,你就真想聽我唱的那些悲曲?我點了點頭。他再次沉重地嘆了口氣,說,你搜集這樣的民歌,是沒有出頭之日的,誰聽這樣的民歌啊。

陳紹純啓開酒,喚我坐在他對面的小方凳上,直接對着瓶嘴飲起酒來。他對我說,他年輕的時候曾經歷過一次死亡,有一天他被一掛受驚的馬車掠倒,送到醫院後,昏迷了二十多天。他說自己甦醒後,耳畔縈繞的就是悽婉的歌聲,那種歌聲特別容易催發人的淚水,從此之後,他就癡迷於這種旋律。那時他是一名中學語文老師,寒暑假一到,他就去鄉村蒐集民歌,整理了很多,還投過稿,但是沒有一首能夠發表。因爲那詞和曲洋溢的氣息都太悲涼了。陳紹純有一個朋友在文化館工作,他曾把民歌拿給他看,他大加讚賞。兩個人聚會時,常常悄悄吟唱那些民歌。文革中,這位朋友揭發了他,說陳紹純專唱資產階級的傷感小調,對社會主義充滿了悲觀情緒,陳紹純開始了挨批生涯。他被打折過腿和肋骨,他們還把他整理的民歌撕成碎屑,勒令他吃下去,讓這頹廢的資產階級的東西變成屎。他就得像一頭忍辱負重的牛一樣,把那些紙屑當草料一樣嚼掉。陳紹純說很奇怪,以前他並不能記住所有的旋律,可它們消亡在他體內後,他卻奇蹟般地恢復了對民歌的記憶,那些歌在他心底生根發芽、鬱鬱蔥蔥,他的內心有如埋藏着一片芳草地,他常在心底歌唱着。只是那些歌詞就像蝴蝶蛻下的羽翼一樣,再也尋覓不到了,所以他的歌是沒有詞的。而那樣的詞在那個年代,就像插在圍牆頂端的碎玻璃屏障一樣,雖然陽光把它們照得五彩斑斕的,但你如果真想貼近它,跨越它,就會被扎得遍體鱗傷。

陳紹純說如果沒有這些歌,他恐怕就熬不到今天了。文革結束後,他又回到學校當教師去了,退休後,就開了深井畫店。他之所以開畫店,就是爲了唱歌方便。家人不允許他在家唱,有一回他唱歌,家裏的花貓跟着流淚。還有一回他唱歌,小孫子正在喝奶,他撇下奶瓶,從那以後就不碰牛奶了,他只得在外面唱歌。

天色越來越暗了,陳紹純的面容在我面前已經模糊了。他對我說,在烏塘,最愛聽他歌的就是蔣百嫂。蔣百失蹤後,蔣百嫂特別愛聽他的歌聲。她從不進店裏聽,而是像狗一樣蹲伏在畫店外,貼着門縫聽。她來聽歌,都是在晚上酒醉之後。有兩回他夜晚唱完了推門,想出去看看月亮,結果發現蔣百嫂依偎在水泥臺階前流淚。

陳紹純的歌聲就是在談話間突然響起來的。他的歌聲一起來,我覺得畫店彷彿升起了一輪月亮,剎那間充滿了光明。那溫柔的悲涼之音如投射到晚秋水面上的月光,絲絲縷縷都洋溢着深情。在這蒼涼而又青春的旋律中,我看見了我的魔術師,他倚門而立,像一棵樹,悄然望着我。沒有巫師作法,可我卻在歌聲中牽住了他的手,這讓我熱淚盈眶。

我回到旅店時,天已經很黑很黑了。週二和週二嫂在吵嘴,原來週二嫂用驢車帶回了一個瘸腿人,此人是個農民,他老婆進城打工,一去兩年,音信皆無。他去尋,發現老婆已跟一家餐館的大廚廝混上了,他跟大廚格鬥,被打折了一條腿。他沒錢醫治腿,又沒錢乘車,就一路拄着拐回他的老家去。週二嫂在站前廣場遇見了這個衣衫襤褸、神情憔悴的人。她就把他扶上驢車,想讓他來旅店睡宿好覺,喝碗熱湯。不料週二對她的義舉大爲不滿,說這個人病得快成灰了,萬一死在店裏,他的家人找來訛上我們,豈不是好心當成了驢肝肺?週二嫂覺得委屈,她說週二,我領回的要是個女人,你就不這麼吹鬍子瞪眼睛的了。週二氣急了,他跺着腳說,你就是領回個天仙,我也只和你睡!

我回到房間,洗了把臉,關了燈,躺在牀上。我的枕畔放着一個電動剃鬚刀盒,這是魔術師的。他在時,我常常在清晨睡意蒙?時,聽到他刮鬍子的聲音。那聲音很像一個農民在開着收割機收割他的麥子。他永別我後,我將他遺落在枕畔的幾根頭髮拾撿起來,珍藏在他變魔術用的手帕中。而這個剃鬚刀槽蓋中,還存着他沒來得及清理的被碾成了齏粉的鬍鬚。我覺得那裏仍然流淌着他的血液,所以也把它珍藏起來。我帶着它出來,就是想讓它跟我一起完成三山湖的旅行。對我而言,它就是一個月光寶盒。我撫摩着它,想着第二天仍然可以到深井畫店傾聽陳紹純的歌聲,便有一種傷感的幸福瀰漫在周身。然而就在那個夜晚,陳紹純永別了這世界沉沉的暗夜,他把那些歌兒也無聲無息地帶走了。

\pagebreak

\section{第五章 沉默的冰山}

我是在凌晨跟週二尋找瘸腿人時,得知陳紹純的死訊的。

週二如以往一樣早起,套上驢來拉磨。他正往磨眼中填泡好的黃豆的時候,爲客人燒洗臉水的週二嫂慌慌張張地闖進磨房,對週二說,不好了,那個腿壞了的人不見了!住店的大都是週二嫂的老客人,譬如運煤的司機,拉腳的小販或是收購藥材的商人,週二嫂就把大家都吆喝起來,幫助她尋找那個失蹤的人。

週二嫂帶着一行人朝西南方向尋找,而我和週二則奔向東北方向。天雖然亮了,但不是那種透徹的亮,街巷中幾乎不見行人,它們灰暗、陳舊得像一堆爛布條。空氣比白天要清爽一些。週二邊尋找邊和我嘟囔,說週二嫂就是這麼個愛管閒事的女人,她要做的事,你若是不依,她倒不和你頻繁地吵鬧,她治理週二的辦法就是在每日的餐桌上只擺上兩碟鹹菜和一盤饅頭。週二在集市混了一天,最惦記的就是晚餐的燒酒和可口小菜,所以他輕易不敢拗着週二嫂行事。他說如果找不回那個人,週二嫂肯定會把醬缸中長了白醭的鹹菜撈出來對付他。我寬慰週二,一個拄着拐的病人,他又能跑多遠呢?諒他是不會出城的。

然而這個人確實消失得無影無蹤了。凡是他能去的地方,比如公交車站、火車站、橋洞、居民區的自行車棚、垃圾箱、公園甚至公廁,我們都找過了。我對週二說,也許週二嫂他們已找回他了,正喝着熱湯呢,於是就折回旅店。豈料週二嫂一行也是失望而歸,這一大早晨撒出去的兩片網均一無所獲,週二嫂淚眼朦朧的。她責備週二,一定是昨晚她和丈夫吵嘴的話被那人聽到了,他一想到男主人不歡迎他,就知趣地在夜半無人注意時悄悄離開。萬一他死在半路上,週二就是殺人兇手。

週二不敢插言,唯唯諾諾聽着。最後他說,他走不遠,我再去找。

我和週二又回到街上。週二說,驢白白拉了磨,今早的豆腐做不成了,這一天的生意算是白搭了,我也去不成集市了。昨天我和謝老鐵下的半盤棋還撂在那兒,想着今天下完,下一步棋該怎麼走我昨晚都想好了,咳!

我寬慰他,沒準一會兒就能找到那人。週二忍不住埋怨道,你說一個大男人,臉皮怎麼就那麼薄啊,聽了兩句難聽的就開溜了,還趁着夜色,真是屬老鼠的,這不是成心要我和老婆鬧彆扭嘛,媽的!

街巷中漸漸有了行人,天也亮了。在主幹街道中,已出現了穿着橘黃背心掃街的環衛工人。我們向她們打聽是否見着一個爬行着的人,她們都搖頭說沒見過。我們走過百貨商場,走過醫院,走過糧油店,從輝來街進入寬成街,又從寬成街插入月樹街。灰濛濛的太陽升起來了,向陽的建築物忍飢受凍了一夜,如今它們吮吸着陽光,看上去光潔而滋潤。車聲起來了,人語也起來了,街市也就有了街市的樣子。我們順着月樹街自然而然來到回陽巷,遠遠的,就見深井畫店不斷有人進進出出。週二對我說,畫店一定出事了,陳老先生從來不這麼早開張,畫店也不會在一大早來這麼多人的。

我們加快了步伐,快接近畫店時,週二碰到一個歪嘴的熟人,他說話有些含混不清,他告訴週二,陳老爺子死了,是讓一幅畫框給砸死的,如今正給他穿壽衣呢。週二拍了一下腿,說,陳老爺子怎麼這麼倒黴!歪嘴人說,聽說他是讓牛枕家的畫框給砸死的,砸到腦殼上了!可能人老了,腦殼跟雞蛋殼一樣酥了,不經砸!歪嘴人說完,擤了一把鼻涕。

沒有陽光跟着我們走進畫店,因爲深井畫店在回陽巷的陰面。有四個人正抻着一塊白布站在櫃檯裏,從裏面傳來聲音。其中一個人低沉地對週二說,別過來,正穿着衣服呢。週二和我就像兩根柱子似的無言地立在那裏了。過了一刻,有一個人直起腰來,是一張老女人的臉,她吩咐那四個撐着白布的人,把白布蒙在陳老爺子身上,看來死者衣裳已經穿好了。幾個人紛紛走出櫃檯,蹲到窗前的一個臉盆裏洗手,彷彿他們剛剛做完一件不潔淨的事似的。洗完手,幾個人直起身來吸菸。週二問那個老女人,顧婆婆,陳老爺子是幾時沒的?顧婆婆深深吸了一口煙,說,今兒一大早我出門潑洗臉水,聽見他家的店門被風吹得嘩嘩響,像是沒閂的樣子,我就過來看看。那門真的沒閂,我進去一看,陳老爺子躺在地上,人早就涼了,他的腦袋旁橫着個畫框,框沒散,玻璃碎了,鑲在裏面的畫也好好的。我認出了那是牛枕他娘要的牡丹。他這是要把畫掛在鉤子上,失手了,把自己給砸死了。顧婆婆又深深地吸了口煙,說,俗話說得真對呀,該着井裏死的,河裏死不了!一個鏡框,要是砸只螞蟻,未見砸得死;砸個大活人竟這麼輕巧,只能說明他該着這麼死麼!

顧婆婆話音才落,牛枕一臉喪氣地進來了。大家見了他都不說話,他也只是反覆說着“這可怎麼好”一句話。顧婆婆吸完那支菸,將菸頭扔掉,進了櫃檯裏面,很快把那張肇事的牡丹圖取了出來。她就像公安人員讓罪犯認證一件血衣一樣,將它攤在地上,對牛枕說,這是不是給你娘畫的?

牛枕抽泣了一下,點了點頭,眼裏淚光點點。

那牡丹圖果然比昨日看上去要鮮豔多了,紅色的紅到了極致,粉色的粉得徹底,看來陳紹純老人已經重新修飾過了這張牡丹圖。顧婆婆又點了一棵煙,對牛枕說,你說鑲着這畫的玻璃碎了不知多少塊,可這張牡丹圖呢,連個劃痕都沒有,真是奇了!

週二見牛枕看着畫的那種哀愁欲絕的表情,就勸慰他說,如果陳老爺子不將畫框懸在房樑下,而是像布店擺放布匹那樣一匹匹地豎在櫃檯上,就不會出這樣的事了。顧婆婆也說,陳老爺子也是怪,畫又不是魚乾肉乾,非要吊起來做什麼,這下好,等於自己捉來個吊死鬼,被小鬼索了性命!

想到那些至純至美的悲涼之音隨着陳紹純離開了這個世界,我流淚了。這張豔俗而輕飄的牡丹圖使我聯想起撞死魔術師的破舊摩托車,它們都在不經意間充當了殺手的角色,劫走了人間最光華的生命。有的時候,生命竟比一張紙還要脆弱。

顧婆婆就是與畫店比鄰的壽衣店的店主,她絮絮叨叨地對大家說,陳老爺子昨夜又唱他的喪曲了,唱了大半宿,她爲了給張順強家扎一對還願用的紙牛紙馬,閉店時快到午夜了,可陳老爺子還在唱歌。顧婆婆還說,她去陳老爺子家報喪時,陳老太婆好似睡着,被叫醒後聽說她男人沒了,一聲都沒哭,反倒打了一個呵欠,說,唱那種歌兒的,有幾個好命的?她的兒孫們聞訊後也不顯得特別悲慼,他們相跟着來到畫店後,還爭論這畫店將來該做什麼。大兒子說要開玩具店,小兒子說要開音像店,沒誰掉眼淚。看他們那架勢,用不上三天,他們就會把陳老爺子推進火葬場。

畫店又涌進來幾個人,他們拿着黑布、挽幛和幾刀燒紙。其中一人的面容酷似陳紹純,看來是他的兒子。顧婆婆問,你們就在畫店佈置靈堂啊?那個像陳老爺子的男子說,唔,我媽說了,不往家拉了,我爸喜歡畫店,就讓他從這兒上路。說完,他從兜裏摸出五十元錢給顧婆婆,說這是賞給她的穿衣錢。顧婆婆顯然對這個錢數不滿,她謝也沒謝,微微撇了一下嘴,將錢掖到褲兜裏,說她店裏沒人照應,如果有事再去叫她,就出了畫店。

我和週二也走出畫店。週二走在前,我在後。我們出門時,牛枕還在哀愁地垂立着,看着那張牡丹圖。週二回頭對我說,看來牛枕今天跟他一樣倒黴,他賣不成豆腐了,牛枕也別想着去集市賣肉了。

由於街巷的寬窄和深度不同,陽光投射下來的影子是不一樣的。有的街道寬闊平坦,街兩側的建築物又低矮,陽光的進入就活潑、流暢,街面上的光影就是明媚而柔和的。但如果是幽長而逼仄的小巷的話,再趕上巷子旁的房屋密集而挺拔,陽光的到來就頗爲吃力,落在巷子中的光影就顯得單薄而陰冷,回陽巷的陽光就是這樣的。走在這樣的小巷中,我越發有一種淒涼的感覺。週二見我失神,就不再回頭與我搭話,他仍然不斷地向行人打聽拄拐人的下落,大家對他的回答總是說不知道。從週二疲塌的步態上,能明顯感受到他的沮喪。

我們回到旅店,週二嫂已經心平氣和地忙着早飯了。原來她碰見了一個運煤的跑長途的司機,他在離烏塘有五六里路的金平莊碰見了一個拄拐的人,他看上去比單腳立着的稻草人還要單薄,金平莊的一個養雞戶正張羅着給他搭便車,讓他回家。週二嫂明白這個倒黴蛋碰上了好心人,心中也就安寧了,對週二的態度也和悅了,問他早餐想吃什麼鹹菜。週二一見週二嫂雲開日朗,連忙回磨房做他的豆腐去了。趕不上上午的集市,他下午去也來得及。

週二嫂告訴我,通往三山湖的火車已經通了,問我什麼時候離開烏塘。我對她說不急。她問我民歌和鬼故事蒐集得怎麼樣了,我便把陳紹純的死訊告訴她。她聽了一驚,說,這老爺子身子骨挺硬朗的,竟然死在一張畫上,這就是命啊。她說他兒子的名字還是陳紹純給取的呢,文革結束後,陳紹純還給上頭寫了信,建議恢復老街巷的名字,回陽巷和月樹街這些一度被廢棄的名字,又重新回到街市中。按週二嫂的說法,陳紹純是烏塘最有文化的人,她說就衝陳紹純給她兒子取了名字的情分上,她一會兒也要買上幾丈白布去弔孝。她還說蔣百嫂要是知道陳老爺子死了,一定會難過的,她喜歡他的歌兒。

週二嫂感受到了我的抑鬱,她說我做的事跟採山貨一樣,山貨的出現是分年份和氣候的,蒐集民歌和鬼故事也是。趕上這個年月聽民歌的人少了,採集起來當然就困難,她勸我不要太難過。她說這兩年蔣百嫂沒少聽陳紹純的歌,她在夜晚酒醉回家後,也常哼上幾曲,估計都是從深井畫店學來的,這樣我完全可以從蔣百嫂那裏挖掘陳紹純掌握的民歌。她的話使我死寂的心又燃起一簇希望之火。不過週二嫂對我講,去蔣百嫂家裏不那麼容易,她早晨起得晚,沒人敢這時敲她的門,她也不喜歡客人去;白天呢,她在集市賣油茶麪;晚上她倒是回家的,但沒個定時,或早或晚,而且如果趕上她喝醉了,帶回家的就不僅是一身酒氣,可能還會有一個男人,這時候更不便打擾她了。

我說沒關係,我可以慢慢等待機會。

週二嫂笑着說,我可不是要拖你的腿,想讓你在我的旅店多住幾天啊。

我哪會那麼想你呢,我說,你對那個沒錢的瘸腿人都那麼好。

一提起瘸腿人,週二嫂又嘆氣了。她說那個人實在可憐,一夜能拐到金平莊,幸虧夜裏沒下雨。不過晚上寒氣大,天又黑,他不知遭了多少罪!說着說着,她的眼睛溼了。她告訴我,烏塘還有一個愛唱歌的人,她專唱婚禮上的歌,叫肖開媚,在城東開了家婚介所。她勸我不妨去見見她,也許她唱的歌對我也有用。

吃過早飯,我就步行到城東去找那家婚介所,還真的好打聽,一找就找到了。不過肖開媚不在,只有一個嗑着瓜子的肥胖女人守在那裏。她對我說,肖開媚今天有活兒,開鞋店的老楊的兒子結婚,她主持婚禮去了。我問肖開媚是否會在婚禮上唱歌,那女人竟然操着一口港臺腔對我說,當然啦,她是去唱喜歌去的啦。烏塘的新媳婦,肖開媚要是不去給唱上幾首喜歌,她們是不會入洞房的啦。她問我是不是也來預約婚禮的,我搖了搖頭,她就興高采烈地說,那你一定是登記找男友的啦,你喜歡醫生嗎,醫生握着手術刀,又掙工資又拿紅包,還不顯山不露水的,安全!我這裏剛剛登記了一個,他老婆得癌了,他讓我先幫他物色着,他老婆是晚期癌症,挺不上幾個月了。你喜歡警察嗎,有個剛離婚的警察,帶着個八歲的男孩,想找一個容貌說得過去的,我看你夠標準啊!她一邊喋喋不休地說着,一邊取來一個花名冊,嘩啦嘩啦地翻着,爲我物色着人選。那一刻我覺得她就是拿着生死簿子的專門勾人魂魄的閻王爺,而我正不知不覺地踏入了地獄之門。從這樣的環境中飛出來的喜歌,肯定透露着銅臭之氣,不會讓人的內心產生真正的喜悅。在我看來,真正的喜悅是透露着悲涼的,而我要尋找的,正是如梨花枝頭的露珠一樣晶瑩的—— 喜悅盡頭的那一縷悲涼!

我失望地離開婚介所,漫無目的地回到街巷中。見到街角有人賣金魚,就湊上去看兩眼;見到一個乞丐從垃圾箱中往出翻騰東西,也湊上去看兩眼。天色有些昏黃,絲絲縷縷的雲彩看上去就像是一片荒草。我進了一家錄像廳,廳裏光線微弱,汗腥味很濃,像是誤闖了魚蝦市場。錄像是循環放映,畫面上是一個女人酥胸半露、同時與兩個男人調情的鏡頭。我看了兩眼,就乏味了,歪在破爛不堪的椅子上睡着了。這一覺竟然睡得比在旅店還要沉迷。等我醒來,電影已轉爲槍戰片,一隊穿迷彩服的士兵與一隊穿便服的人在叢林中激戰正酣,噠噠噠的槍聲和火光交替出現。我覺得肚子餓了,晃晃悠悠地步出錄像廳,一看手錶,已是午後一時了,便就近踅進一家小吃店,要了一碗米飯,一盤地三鮮。在等菜的時候,聽見兩個面色黎黑的食客在議論剛剛發生的一件事情。說是那個唱喜歌的肖開媚今天上午主持鞋店老楊的兒子的婚禮時,被礦工劉井發給打了。肖開媚介紹了一個外鄉來的女子給這礦工,誰也不知道她是來烏塘“嫁死的”。劉井發和她過了兩年,總不見她懷孕,讓她去看病吧,這小媳婦反而污衊劉井發,說他的種子不好使。劉井發起了疑心,砸開了小媳婦終日上着鎖的箱子,結果發現了好幾張關於他的人身意外傷害保險單,劉井發將她暴打一頓,要休了她,小媳婦倒也不在乎,她說自己結婚前就戴了環,根本就沒想給他生個一男半女的。劉井發認爲婚介所的肖開媚一定是和小媳婦串通好了,介紹了這麼個毒蠍女人給他,就揣上一把斧頭,鬧了老楊兒子的婚禮,在肖開媚的背上砍了十幾斧子。如今肖開媚被拉進醫院急救,劉井發被警車帶走,攪得婚禮沒點喜慶的氣氛,老楊哀嘆自己賣鞋招來了“邪氣”,連新媳婦敬的喜酒都不吃了。

咳,你說這新媳婦帶着個環和人家結婚,等於往肚子裏放了一張網,那劉井發撒下的魚苗再好,也是個被擒的命!其中那個長着對招風耳的食客說。

另一個吃東西時發出響亮吧唧聲的食客說,我要是娶了這樣的媳婦,就把她捆上,讓她天天跪在門檻上,每隔五分鐘喊我一聲“爺爺”,不喊就揍,我就不信弄不服帖她!他進而分析煤礦事故多的原因,那是由於地下是閻王爺居住的地方,活人天天下去採煤,等於掘閻王爺的房子,讓他不得安生,他當然要大筆一揮,取出生死簿子,把那些本不該壯年死去的人的名字一一勾上,提早帶走他們。所以死在井下的礦工,總是三五成羣。

招風耳說,現在行了,下井的一班是九個人,上頭不是有文件嗎,超過十人以上的死亡事故才上報,死九個人,等於是白死!

王書記也真是命好,小鷹嶺煤礦那次事故,要是蔣百也在井下,剛好是十個人,一上報他就得倒黴,還不得來個行政記大過處分?哪有日後被提拔的份兒!媽的,蔣百也真是甜和他!你說蔣百究竟去哪兒了,我估摸着他那天還是下井了,只不過沒找到屍首罷了。不然他家的狗怎麼天天還是去汽礦站迎他?狗從哪兒把人送走,自然是在哪兒等主人回來的!

他們接着慨嘆被不明不白拋棄了的蔣百嫂,慨嘆糊里糊塗沒了爹的蔣三生,慨嘆採煤不是人乾的活兒。本來他們的飯已吃完了,慨嘆來慨嘆去,他們覺得世事難料,就說不如趁着休班,一醉方休,明天下了井,能不能回來,還兩說着呢。我這才明白,他們也是礦工,難怪他們的臉那麼黑呢,好像每一道皺紋裏都淤積着煤渣。他們要了一斤燒酒,兩個小菜,開始了新一輪的吃喝。在這種時刻,我也特別想喝上一點酒。我吆喝來店主,要他爲我拿一壺酒,添上一碟五香花生米和一碟鹹魚。店主吃驚地看着我,半晌沒有反應過來,他大約沒有見過一個女人會來這裏要酒喝,所以當他朝竈房走去的時候,不由自主地嘟囔道:又一個蔣百嫂——

兩個礦工無所顧忌地聊着天,他們一會兒講鄰里間的事兒,一會兒又講親戚間的事兒和夫妻間牀上的事兒,非常地放縱,又非常地快樂。我呢,對着幾碟小菜獨斟獨酌着。小吃店的衛生狀況很差,蒼蠅絡繹不絕地在杯盤碗盞間飛起落下,趕都趕不及,只好對它們聽之任之,也算有生靈陪着我這孤獨的酒客。

時光在飲酒的過程中悄然流逝了。裹挾在酒中的時光,有如斷了線的珠子,一粒粒走得飛快。不知不覺間,天色已暗淡了,那兩個礦工是什麼時候走的我竟一無所知。我飄搖着向外走的時候,店主吆喝住了我,說,哎,你還沒付賬呢!看來我把這小吃店當成了自己的家。我掏錢買單的時候,店主問我,你不是烏塘人吧?我點了點頭。店主把零錢找還我的時候,說,世上沒有趟不過去的河,遇事想開點!

我覺得自己輕飄得就像一片雲。如果我真是一片雲就好了,我能飛到天上,看看我的魔術師是否在雲層背後、手持魔杖對我微笑?我叫了一輛人力三輪車回旅店。路過暖腸酒館時,我看見了蔣百嫂的背影,她一定又去吃酒了。而她家的狗,正在路邊有氣無力地啃着一簇野草。

我回到房間倒頭便睡,一條波光盪漾的大河出現在夢中。我站在此岸,望着對岸的青山,忽然看見一隻鷹從青山中飛起。我的目光追隨着這隻鷹,它突然就幻化爲一朵蓮花形態的彩雲;當我對着這雲的嫺雅之美而驚歎不已時,彩雲又變爲一隻鹿,讓人覺得天上也有叢林,不然這鹿緣何而生?正當我想要仔細察看鹿身後的天空是否有叢林時,它卻變幻爲一條搖頭擺尾的魚。而天空下面的青山,卻依然是青山。我對着青山冥想之時,一陣哭鬧聲撕裂了我的夢境。睜眼一看,天已黑了,去拉燈,燈卻依然黑着臉,像是與什麼人生了氣,不肯綻放笑容。我摸黑走出房間,見走廊盡頭有一支蠟燭坐在花盆架上,它勃勃燃燒着,投下一帶顫動的乳黃的光影。這光影於我來講彷彿是一片片凋零的落葉,我小心翼翼地踩着它走過,踩出了一腳的蒼涼。

正當我要走出屋子,想看看外面究竟發生了什麼事時,背後傳來了腳步聲,回頭一望,原來是週二擎着一盞油燈從磨房走了過來,他大概剛泡完豆子。黃豆不被泡軟,是上不了磨盤,做不成豆腐的。

我問週二是誰在外面哭鬧,聽上去撕心裂肺的,怪人的。週二嘆了一口氣,說,能是誰啊?是蔣百嫂!她醉了,又趕上停電,她就鬧,非說要用炸藥包把供電局給崩了!

週二對我說,蔣百失蹤後,蔣百嫂似乎特別怕黑暗,逢到停電的時刻,她就跟瘋了似的四處奔走呼號,絕不肯在家裏呆一刻。週二嫂爲此買了很多包蠟燭送她,可是她並不喜歡燭光,嫌它身上不帶電。給她送油燈呢,她非說油燈睜的是鬼眼,不懷好意地看她。週二嫂就買來一盞電瓶燈送她。按理說電瓶燈發出的光與電沒什麼區別,可蔣百嫂仍是嫌棄它,說它把電藏在自己的肚子中,不能傳輸給別的電器,是個廢物。鄰居們都知道蔣百嫂受不了沒電的時光,所以一遇停電,週二嫂不管手上忙着什麼緊要活兒,都要立馬放下,去安慰蔣百嫂。蔣百嫂在停電時刻暴躁不安,而一旦室內電燈復明,她就奇蹟般地安靜下來了。

週二把油燈擺在門口的鞋櫃上,陪我出去看蔣百嫂。街面上沒有車輛駛過,也沒有行人,路燈一律黑着臉,只有兩束銳利的手電筒光在蔣百嫂身上閃來閃去,使她看上去像個站在水銀燈下拍夜景戲的演員。

週二嫂說,你回屋吧,蔣百嫂,夜裏涼,你要是感冒了,誰心疼你啊?你回了屋,電也就來了。

蔣百嫂跺着腳哭叫着,我要電!我要電!這世道還有沒有公平啊,讓我一個女人呆在黑暗中!我要電,我要電啊!這世上的夜晚怎麼這麼黑啊!!蔣百嫂悲痛欲絕,咒罵一個產煤的地方竟然還會經常停電,那些礦工出生入死掘出的煤爲什麼不讓它們發光,送電的人還有沒有良心啊。

我從未見過一個女人爲了爭取光明而如此激憤,而這光明又必須是由電而生的,這讓我困惑不已。蔣百嫂哭叫着,週二嫂和另外兩名婦女則好言勸解着,打算把她架回屋子,可她像頭被激怒的公牛一樣,沒有回去的意思,不斷地往前掙,聲言要買兩噸炸藥,把供電局炸成一片廢墟。正當大家一籌莫展之際,路燈就像長了腿似地跳了一下,電閃閃爍爍地來了。蔣百嫂打了個激靈,立刻安靜下來了。

路燈亮了,居民區的燈也亮了。光明中蔣百嫂雖然也是一臉的悲涼,但她已恢復了理智。她對週二嫂等人說着對不起,然後領着一直在旁邊打着哆嗦的蔣三生回家。

蔣百嫂走後,我隨着週二和週二嫂回旅店。週二一進門就奔向油燈和燭臺,忙不迭地“噗噗”將它們吹滅。週二嫂說,蔣百嫂確實怪,一停電就跟瘋了似的,任誰也勸阻不了,除非是電回來了,她才恢復平靜。我覺得這其中一定隱藏着什麼祕密。週二說,能有什麼祕密呢,男人就是女人的電,缺不了的;離了這個電,再好的女人也乾枯了!說着,十分自得地衝週二嫂擠着眼睛,似乎在提醒她,她身上的活力是他賦予的。週二嫂“呸”了週二一口,說,餵你的驢去吧,要不它明天早晨哪有力氣拉磨!週二哼着小曲,樂陶陶地去磨房了。

在這樣一個夜涼如水的夜晚,我特別想和蔣百嫂聊聊天。我沒有徵求週二嫂的意見,獨自出了旅店,走進一家食雜店,買了兩瓶二鍋頭,一包花生米、一袋醬雞爪以及幾個松花蛋,敲蔣百嫂家的門去了。

蔣百嫂的家門外掛着一盞燈,還吊着一串風鈴,所以輕輕敲幾下門,風鈴就會跟着鳴響。那風鈴很別緻,一隻彩色的鐵蝴蝶下吊着四串鈴鐺,它們發出的聲音非常清脆,看來蔣百嫂把它當門鈴來用了。

開門的不是蔣百嫂,而是蔣三生。他見了我有些躲躲閃閃的。我問他,你媽在家嗎?他先是說在,接着又說沒在。他好像剛哭過,臉上的淚痕隱約可見。他立在那裏,像個小門神,沒有讓我進屋的意思。

我認定蔣百嫂就在屋裏,就說要進屋等她。蔣三生畢竟是個不諳世事的孩子,他噔噔地跑到一扇屋門前,說,是在周媽媽家住店的人,我說了你不在,可她還要進來等你!

我已經不請自進地跨進門檻了。一股香氣撲鼻而來,是幽微的檀香氣味,看來蔣百嫂在焚香。屋子素樸而整潔,陳設看上去規矩、得體,與我事先想像的零亂情景大不相同。有一點讓我覺得奇怪,明明有兩扇屋門,進門的小廳裏卻擺着一張小牀,一看就是蔣三生的,蔣百嫂爲什麼不讓他住在屋子裏呢?

我把酒菜放在小廳的圓桌上。蔣百嫂推開一扇藍漆門,提着一把黑沉沉的大鎖頭,赤紅着臉走出來,反身把門鎖上。她再次轉過身來時連打了幾個寒戰,好像她剛從冰窖中出來。也許是剛纔這一場哭鬧消耗了她太多氣力的緣故,她看上去有些疲憊,髮髻也鬆垂了,幾綹髮絲像樹杈那樣斜伸出來,而她的脣角,漾着一點紅,想必先前她暴怒之時不慎咬破了它。她有些木然地面對着我,久久無話,只是不斷地伸出舌頭舔拭脣角,微蹙着眉。那血跡被吸乾後,慢慢地又洇了出來,好像她的脣角是個火山噴發口,金紅的熔岩要不斷涌現。

你找我有事麼?蔣百嫂哀哀地看着我。

那天我來烏塘,在暖腸酒館,你邀我喝酒,我不識相,今天特地帶了酒來,想和你喝上幾盅,說說話,也算賠罪了。我看着她背後那扇上了鎖頭的門說。我從沒見過一個人在自家屋內還得上鎖,那裏一定隱藏着祕密。

我聽週二嫂說,你是來蒐集鬼故事和民歌的。蔣百嫂吁了一口氣對我說,我不會說鬼,更不會唱民歌。

今晚我不想聽鬼故事,更不想聽民歌,我說,我只想跟你喝酒。我盯着她滿懷哀愁的眼睛,說,今天晚上太冷太冷了。說完這話,我確實覺得寒冷,忍不住打了一個哆嗦。

那好吧。蔣百嫂指着桌子上我帶來的酒菜說,廳裏涼,去我的屋裏喝吧。她吩咐蔣三生把我帶來的東西拿到裏屋的地桌上。蔣三生答應着,麻利地將酒菜兜在懷裏,奔向裏屋,那樣子活像一個甩着長尾巴的小松鼠抱着鬆塔快樂地前行。

檀香的氣息越來越濃了,我故做輕描淡寫地對蔣百嫂說,從那屋裏飄出來的香氣可真好聞啊,我在佛誕日常去寺廟燒香,聞到的就是這種氣味。

蔣百嫂淡淡地說,那裏面供着祖宗的牌位,所以時常要上上香,說完,她率先朝屋裏走去。

在跟着蔣百嫂朝屋裏走去的時候,我在她身後悄悄貼近那扇藍門,我聽見一陣“嗡嗡”的轟鳴聲,好像裏面有什麼機器在工作,這更令我疑惑重重。供奉祖宗,環境應該是清淨的,爲什麼還會有這樣的聲音發出?

蔣百嫂的屋子也是整潔的,屋子的佈置以藍印花布爲主,比如窗簾、牀單、縫紉機以及電視機上,掛的、鋪的、苫的都是藍印花布,看上去素雅而美觀。我很難想像蔣百嫂會在這樣的屋子裏和形形色色的男人鬼混。

蔣三生已經把吃食搬到窗前的桌子上了。那是一張一米見方的方桌,左右各擺着一把椅子,桌上放着兩雙筷子,兩個白瓷酒盅,還有半瓶喝剩的酒、一袋青豆以及半袋牛肉乾。看來蔣百嫂常在這裏邀人同飲。

三生,你睡去吧,沒你的事了。蔣百嫂說。

蔣三生答應着,乖乖回到門廳去了。

我問蔣百嫂,怎麼給兒子取了這麼個名字,聽上去老氣橫秋的。

蔣百嫂說,我頭一胎流產了,流下的是對雙胞胎,照算命人的說法,我算是有過兩個孩子了,他出生,排行就是老三了,當然得叫他三生了。

哦,流了產的孩子也算數啊,我說。

那不也是從自己身上掉下來的肉麼,當然算數了。蔣百嫂問我,你有孩子嗎?

我搖搖頭。

蔣百嫂問,你沒結婚?要不是你不會養活?再不就是你男人不行?

我笑了,說,都不是。停頓了一刻,我告訴她,我正想要孩子的時候,我愛人離開了我,他不久前去世了。

蔣百嫂嘆息了一聲,哀憐地看了我一眼,說,咱姐倆原來是一個命啊。

我心中想,難道蔣百並不是失蹤,而是死了?

蔣百嫂大概意識到失言了,她將我讓到椅子上,說,我男人失蹤了快兩年了,沒有一點音信,我這不也等於守活寡麼?

見我沒有附和,她又機智地引入先前的話題,說她懷的那對雙胞胎之所以流產,是被丈夫給嚇的。那年礦上發生透水事故,蔣百那天也下井去了,聽到消息後,她認定蔣百已別她而去,一陣哭嚎,不想動了胎氣,白白葬送了一對雙胞胎的性命。其實那天出事的現場,並不在蔣百的作業點。蔣百安然無恙地回來了,可她的肚子卻像一片破網似地癟了。她慨嘆做礦工的孕婦,肚裏的孩子隨時可能成爲遺腹子。

蔣百嫂坐下來,她家的電話響了。電話被蒙在牀單下,鈴聲乍響時,感覺牀下有個妖怪在叫,嚇了我一跳。蔣百嫂撩開牀單接起電話,喂了一聲,有些不耐煩地說,我在集市站了一天,腰疼,閂門睡了!說着,氣咻咻地擱下聽筒。我猜這或許是哪個男人想來這裏討便宜,反倒討了個沒趣。

蔣百嫂坐到我對面的椅子上,啓開酒對我說,要是誠心跟我喝,得連幹三盅。我答應了。她熟稔地斟酒,瓷盅裏的酒盪漾着,不能再多一滴,也不能再少一滴的樣子。三盅酒落肚,只覺得從口腔直至肚腹有一條火光在寂靜地燃燒,身上熱乎乎的,分外舒展。蔣百嫂指着我的臉笑着說,這世上愛塗胭脂的人真是傻啊,酒可不就是最好的胭脂麼!你瞧你,一喝上酒,黃臉就成了桃花臉,要多好看有多好看!

一喝上酒,我們就比先前顯得親密了。她問我,你男人是幹什麼的?怎麼死的?我一一對她說了,蔣百嫂挑着眼角說,魔術師不就是變戲法的麼?你嫁個變戲法的,等於把自己裝在了魔術盒子裏,命運多變是自然的了!

我是一個不願意在人前流淚的女人,但在蔣百嫂面前,我淚水橫流,因爲我知道她的心底也流淌着淚水。蔣百嫂一盅一盅地斟着酒,我一盅一盅地啜飲着,我就是一堆冰冷的乾柴,而這如火苗一樣的酒,又把我燃燒起來。我絮絮叨叨地敘述魔術師離開我後,我怎樣一次次在家裏痛哭,怕驚擾了鄰居,我就跑到衛生間,打開水龍頭,將臉貼近它,讓我的淚水和着清水而去,讓我的哭聲融入嘩嘩的水流中。我還講了魔術師的葬禮,來了多少人,別人送的花圈又如何被我清理出去,甚至他將被推進火化爐前,我對他最後的乞求,乞求他把自己變活,以及我留在他冰冷的額頭上的最後一個熱吻,都對她毫無保留地傾訴了。很奇怪,蔣百嫂對我的這番話並沒有抱之以同情,相反倒是一陣接着一陣的冷笑,好像我的哀傷不足掛齒,她這種冰冷的態度讓我不寒而慄!

蔣百嫂沉默着,她啓開另一瓶酒,兀自連幹三盅,她的呼吸急促了,胸脯劇烈起伏着,她突然“哇——”地一聲大哭起來,說,你家這個變戲法的死得多麼隆重啊,你還有什麼好傷心的呢!他的朋友們能給他送葬,你還能最後親親他,你連別人送他的花圈都不要,燒包啊,有的人死了也燒包啊。你知不知道,有的人死了,沒有葬禮,也沒有墓地,比狗還不如!狗有的時候死了,疼愛它的主人還要拖它到城外,挖個坑埋了它;有的人呢,他死了卻是連土都入不了啊!

她這番話使我聯想到蔣百,難道蔣百已經死了?難道死了的蔣百沒有入土?不然她何至於如此哀慟?

蔣百嫂徹底醉了,她一會兒哭,一會兒笑,一會兒訴說。她拍着桌子對我說,烏塘的領導最怕的是她,如果她想把領導從官椅上拉下來,那就跟碾死一隻螞蟻一樣容易。他們現在戴的是烏紗帽,可只要我蔣百嫂樂意,有一天這烏紗帽就會變成孝帽子!

蔣百嫂唱了起來,她唱的歌與陳紹純的一樣,是哀愁的旋律。不過那歌裏有詞,而歌詞反反覆覆只是一句:這世上的夜晚啊——,聽得我內心彷彿奔涌着蒼涼而清幽的河水。她唱累了,搖搖晃晃地撲到牀上,睡了。是午夜時分了,我毫無睡意,只是覺得頭暈,如在雲中。

蔣百嫂哼着翻了一下身,她的黑色棉線衫褪了上去,露出了腰肢,我看見她的腰帶上拴着一把黃銅大鑰匙,我認定它屬於那扇上了鎖的藍漆屋門的,便悄悄走上前,取下那把鑰匙。

我掂着那把鑰匙走出去,小廳的燈關了,看來蔣三生已經睡了,依稀可見小牀上蜷着個小小的人影。我鎮定一番,打開那把鎖,推開屋門。撲向我的是檀香氣和光影,屋子吊着盞低照度的燈,它像一隻蔫軟的梨一樣,散發出昏黃的光。這屋子只有七八平方米,沒有牀,沒有桌椅,四壁雪白,拉得嚴嚴實實的窗簾也是雪白的,有一種肅穆的氣氛。北牆下襬着一臺又高又寬的白色冰櫃,冰櫃蓋上放着一隻香爐,一盒火柴、一包檀香以及供奉着的一盤水果。冰櫃的壓縮機正在工作,轟鳴聲在寂靜的夜裏聽上去像是一聲連着一聲的沉重的嘆息,我明白先前聽到的嗡嗡聲就是這個大冰櫃發出來的。蔣百嫂爲什麼會在冰櫃上焚香祭祖,而卻不見她祖宗的牌位?我覺得祕密一定藏在冰櫃裏。我將冰櫃上的東西一一挪到窗臺上,掀起冰櫃蓋。一團白色的寒氣迷霧般飛旋而出,待寒氣散盡,我看到了真正的地獄情景:一個面容被嚴重損毀的男人蜷腿坐在裏面,他雙臂交織,微垂着頭,膝蓋上放着一頂黃色礦帽,似在沉思。他的那身藍布衣裳,已掛了一層濃霜,而他的頭髮上,也落滿霜雪,好像一個端坐在冰山腳下的人。不用說,他就是蔣百了。我終於明白蔣百嫂爲什麼會在停電時歇斯底里,蔣三生爲什麼喜歡在屋頂望天。我也明白了烏塘那被提拔了的領導爲什麼會懼怕蔣百嫂,一定是因爲蔣百以這種特殊的失蹤方式換取了他們升官進爵的階梯,蔣百不被認定爲死亡的第十人,這次事故就可以不上報,就可大事化小。而蔣百嫂一定是私下獲得了鉅額賠償,纔會同意她丈夫以這種方式作爲他生命的最終歸宿。他沒有葬禮,沒有墓地。他雖然坐在家中,但他感受的卻不是溫暖。難怪蔣百嫂那麼懼怕夜晚,難怪她逢酒必醉,難怪她要找那麼多的男人來糟踐她。有這樣一座冰山的存在,她永遠不會感受到溫暖,她的生活註定是永無終結的漫漫長夜了。

我悄悄將冰櫃蓋落下來,再把香爐、火柴、果盤一一擺上去。我鎖上門,把鑰匙拴回蔣百嫂的腰帶上,走出她的家門。這種時刻,我是多麼想抱着那條一直在外面流浪着的、尋找着蔣百的狗啊,它註定要在永遠的尋覓中終此一生了。我很想哭,可是胃裏卻翻江倒海的,那些吞食的酒菜如污泥濁水一般一陣陣地上涌,我大口大口地嘔吐着。烏塘的夜色那麼混沌,沒有月亮,也沒有星星,街面上路燈投下的光影是那麼的單調和稀薄,有如被連綿的秋雨漚爛了的幾片黃葉。我打了一串寒戰,告訴自己這是離開烏塘的時刻了。

\pagebreak

\section{第六章 永別於清流}

我已經把臉塗上厚厚的泥巴,坐在紅泥泉邊,沒人能看見我的哀傷了。比之烏塘,三山湖的陽光可說是來自天堂的陽光,清澈雪亮如泉水。塗了泥巴的身體被曬得微微發熱,我覺得自己就是一塊被放到大自然中等待焙制的麪包,陽光用它的文火,絲絲縷縷地烤炙着我。泉邊坐着一些如我一樣渾身塗滿了泥巴的人,他們也在享受陽光和清風,我無法看見他們臉上的表情,大家臉上的表情,都被那濃雲一樣密佈的泥巴給遮蔽了,所以我不知道他們是哀愁呢還是快樂。

原來的紅泥泉被劃分爲兩個區域,男女各半,只要望見一羣塗了泥巴的人中青煙繚繞着,那一定是男人所在的地方,這羣泥人喜歡手裏夾着香菸,邊抽邊享受陽光。後來紅泥泉的生意不如其他的溫泉,經營者分析這是把男女分開的緣故,於是兩個區域又合二爲一,男男女女可以混雜在一起。果然,生意又漸漸回潮。原來之所以將男女分開,是由於許多男賓客連短褲都不穿,說是泥巴已將禾幺.處嚴嚴實實裹上,短褲實在是多餘。而一些隨意的女賓客,也喜歡裸露着乳防。男女混雜之後,規定是入紅泥泉的客人必須要穿背心和短褲,但違規者大有人在,經營者權當看不見,聽之任之。其實柔軟的紅泥已經是上帝賜予人類最好的遮羞布,客人的選擇不是沒有道理的。一羣泥人坐在紅泥泉邊的情景,讓我聯想到上帝造人的情形。這種能治療很多疾病的紅泥,淤積在碧藍的湖水深處,柔軟細膩,一觸摸便知是經過了造物主千萬次的打磨、淘洗,又經過了千百年和風細雨的滋潤,才釀得如此的好泥。

坐在泉邊的,有許多對戀人。雖然身裹泥巴不方便講話,但從他們手拉手的舉止上,完全能感受到他們的脈脈深情。情侶們的目光,也就跟這光芒四射的陽光一樣,火辣辣的。我是多麼的羨慕這樣的目光啊。如果魔術師坐在我身邊,他也會拉着我的手的,可他卻被一頭跛足驢給接走了。我在心底輕輕呼喚他的名字,淚水奔涌而出。淚水使臉上的紅泥更加潤澤,融入紅泥的淚水已經被調化爲最養顏的膏脂了。

我通常上午時將通身塗滿泥巴,坐在紅泥泉邊釋放淚水,午後再去真正的溫泉浸泡一兩個小時。從溫泉出來,換上便裝,即可一身清爽地在三山湖景區閒走。

我喜歡逛賣火山石的攤牀。那些火山石形態不一,被開發出的產品也就各不相同。那些嶙峋崢嶸的因其妖嬈之氣而被做爲盆景;細膩光滑的則被鑿成筆筒和首飾盒;而紋理如蜂窩一樣粗糙的,十有八九被當做了磨腳石。在賣磨腳石的攤牀前,我遇見了一個七八歲左右的男孩,與其他赤膊、光頭的男孩不同,他戴一頂寬檐草帽,穿着長袖衫,長褲,袖筒寬大,而且衣着的顏色是藏青色的,看上去老氣橫秋,他袒露於臉上的笑容,便有一種受擠壓的感覺。他在攤牀前招攬生意,而進行交易的,是一個面色黎黑的站在少年身後的獨臂男人。男孩不像其他的生意人,採取的是花言巧語的吆喝或是圍追堵截的兜售,他用變戲法的辦法引起遊客的注意。只見他手裏握着一枚溫泉煮蛋,把玩片刻後,這雞蛋忽然幻化爲一塊磨腳石,當遊人對着磨腳石驚歎不已時,他又把雞蛋飛快地變回掌心中。遊人喜愛這男孩,就是不買磨腳石,也要買上兩枚雞蛋,清瘦的獨臂人的生意也就比其他賣火山石的攤牀要好得多了。

經過攤牀的次數多了,我知道獨臂人姓張,男孩叫雲領,他們是一對父子。因爲其他的生意人跟他們說話時,對獨臂人愛說,老張,你行啊,你家雲領在前面變戲法,你後面收着銀子!而對男孩說的則是,雲領,你這小東西這麼會變戲法,在三山湖可惜了,你該進大城市去!當然,也有人用鄙夷的目光瞟着男孩,撇着嘴說,手腳這麼快,別出落成個賊!

雲領變的戲法,明眼人能一眼望穿,他的那兩條腕口緊束的寬大袖筒,因爲預先放置了雞蛋和磨腳石,沉甸甸地下垂着,彷彿裏面藏着貓。但我喜歡看他帶着一股大人的神色展覽他的招數,他能讓我想起魔術師。我三番五次地去,接二連三地買磨腳石,旅館房間的旅行袋中,聚集了太多的火山石,好像我是個採集礦石標本的考古學家。

有一個下午,我又去了雲領家的攤牀。他顯然對我已熟識了,見了我脣角浮出一縷笑容。那笑容很像晚秋原野上的最後的菊花,是那種清冷的明麗。我帶了一條五彩絲線,先向他展示那絲線的完整,然後將它輕輕抖摟一下,絲線就斷爲兩截了;當雲領目瞪口呆時,我輕輕倒一下手,絲線又連綴到了一起。雲領嚥了一口唾沫,回身看了一眼父親,很無助的樣子。獨臂人警覺地看着我,拈起一塊磨腳石對我說,你天天來我家的攤位,這個白送給你,算是我的一點心意。我接過火山石,掂了掂,把它又還給獨臂人。

雲領不再變戲法了,他定定地盯着我,問我怎麼也會幹這個。好像我搶了他的飯碗,他的神情中帶着濃濃的委屈和隱約的憤怒。我想告訴他一個魔術師的妻子做這點小把戲算不得什麼,可我沒有說。我鼓勵沮喪的雲領接着做生意,我不過是想逗逗他玩而已。獨臂人這纔對我和顏悅色,他送給我兩枚泉水煮蛋。我拿着雞蛋剛散步到另一個賣火山石的攤牀前,雲領追了過來,氣喘吁吁地站在我面前,什麼也不說,滿懷乞求的樣子。我問他,你爸爸讓你討要這兩隻雞蛋的錢?他搖了搖頭。我又問,你想讓我再買幾塊磨腳石?他依舊搖了搖頭。他猶豫了許久,才吞吞吐吐地問我住在哪座旅館,說他散了攤兒後想去找我。我笑了,問,你想跟我學魔術?他的眼睛立刻就溼潤了,他急切地問,你真的是魔術師?我笑着搖搖頭,他似乎有些失望。不過當我告訴他我住的旅館的名字和房間號碼時,他還是顯出熱情,我說完後,他重複了兩遍,以求記牢。

夜幕降臨,泡溫泉的人少了,去娛樂的人多了。三山湖景區的咖啡屋、餐館、酒吧、按摩屋、歌廳、檯球室和保齡球館燈影燦爛、人聲鼎沸。在景區的西北角,聚集着一羣放焰火的遊客。大多的遊客來自禁放焰火的大都市,所以三山湖設置了這樣一個自由放焰火的娛樂項目,深受遊客喜愛。夜幕如一塊巨大的沉重的畫布,而在半空中明媚升騰變幻着的焰火則如滴滴油彩,將這塊本無生氣的畫布點染得一派絢麗,歡呼聲和着焰火的妖嬈綻放陣陣響起。我遠遠地看了會兒焰火,就回客房等待雲領。

雲領不是自己來的,當敲門聲響起,我打開房門後,發現站在昏暗走廊裏的,還有獨臂人。他們見了我並不說話,只是笑着。大人和孩子的笑都不是發自內心的,所以那幾團笑容讓我有望見陰雲的感覺。我將他們讓進屋門。

雲領的裝束與白天一模一樣,連草帽還戴在頭上,看來這草帽並不是爲了遮陽的。而獨臂人則換下了白汗衫和藍褲子,穿上了一套黃綠色的套裝,這使瘦削的他看上去格外像一株已經枯黃了的草。雲領比獨臂人顯得要大方一些,他不請自坐在窗前的沙發上,還欠着屁股顛了幾下,大約在試探沙發的彈性。已經被無數客人壓迫得老朽的沙發,發出喑啞的叫聲。獨臂人呢,他大約覺得沙發是奢侈品,他打量了它半晌,最後還是坐在了梳妝鏡前的一把硬木椅子上,而且坐得很端正。我倒了兩杯白水分別遞給他們,獨臂人慌張地站了起來,連連說他不渴,將水接過來後放在了梳妝檯上;雲領呢,他痛快地接過杯子,託在掌心旋轉着,問我,你能把白水變成紅水嗎?我說不能。雲領笑着說我能,他的手抖了一下,那杯水就是紅色的了,不知他眼疾手快地往水裏投了什麼顏料。獨臂人訓斥兒子,雲領,你不是來學習的嗎?怎麼這麼不謙虛,白白糟踐了一杯水!雲領說,這是食用色素,藥不死人,怎麼就不能喝呢!說完,咕嘟咕嘟地將那杯水一飲而盡。

獨臂人呵斥雲領的那番話,已經讓我明白他們來這裏的意圖了。果然,獨臂人懇求我,希望我能教雲領幾套新的招數,因爲他下午時見我能把五彩絲線斷了又連接上,一看就身手不凡,是大地方來的魔術師。而云領會的招數,客人已經不覺得新鮮了。說完,他用那唯一的手從褲兜裏掏出一百元錢,將它放在梳妝檯上,說,就當是學費了,你別嫌少,你要是願意,明兒再去我的攤子拿幾塊磨腳石!

到了這種時刻,我只能如實告訴他,我只會這點小把戲,真正懂魔術的是我丈夫,可他不久前去世了。獨臂人“啊啊”地叫了兩聲,說着對不起,我沒有想到會是這樣。他繼而問我,魔術師是怎麼死的?我告訴他是一輛破爛不堪的摩托車撞死了他。獨臂人嘆了一口氣,說,這就是命啊,像雲領他媽,一條小狗就要了她的命!

獨臂人對我說,以前他和妻子一直在三山湖景區做工,他爲客人放焰火,妻子則受僱在髮廊工作,她剃頭剃得好。來三山湖度假的都是些有錢人,他們不僅帶着情人來,有的還抱來自家的寵物,非貓既狗。那些狗沒有個頭大的,一個個嬌小玲瓏,有的頭上還扎着蝴蝶結,拾掇得比小女孩都漂亮。有一天,髮廊來了一個抱着小狗的女賓客,雲領他媽給她剪頭髮時,它還安安靜靜地呆在主人懷裏,可當她爲客人噴摩絲時,小狗以爲主人受到了威脅,跳起來咬了雲領他媽的手,把手背給咬破了。女賓客倒也不是個吝嗇的主兒,拿出二百塊錢,讓雲領他媽去打狂犬疫苗。髮廊的老闆娘對雲領他媽說,一隻小狗,天天又洗澡,比人都乾淨,能有什麼病菌啊,這錢不如分了算了。於是,老闆娘留下一百,雲領他媽拿回一百,覺得撿了個大便宜。那傷口好得很快,結痂後又長了新皮,可是幾個月後,妻子突然間變了個人似的,她整天暴躁不安,常常和客人大吵大鬧,只要拿起剪刀,想的就是給客人剃光頭,老闆娘辭退了她。原想着她回到家後就會安靜了,可她照例鬧個不休,她最不能看見水,一見了水就會哆嗦在牆角。家人把她送到醫院,診斷是患了狂犬病,沒有多久,人就死了。獨臂人說到這兒,聲音哽咽了,雲領大約也跟着難受了,他說要撒泡尿,跑到衛生間去了。

獨臂人說,雲領很忌諱別人說他媽媽死了,他總說她去了另外的地方了。他從不去媽媽的墳上,說是媽媽沒有呆在土裏。這兩年陰曆七月十五的夜晚,他總是提着一盞河燈獨自出門,說是單獨去會他的媽媽,別人不能跟着。他去哪裏放河燈,連他這個做父親的都不知道。想必他走了很遠很遠的路,因爲他回來時,總是午夜時分。獨臂人說,後天又是七月十五了,雲領那天晚上又得出門了。咳,我真不放心他一個人走夜路。

雲領從衛生間出來了,他紅着眼圈,似乎剛剛偷偷哭過,可臉上卻做出無所謂的表情,他聳着肩,抱怨這家旅館的衛生間小,沒有其他湖畔山莊的大,做出一副見多識廣的樣子。我問他爲什麼晚上還要戴着草帽,他此時露出了真正屬於兒童的天真笑容,說,我尋思你能教我變戲法呢,你看——

雲領摘下草帽,只見草帽的底部嵌着個鑲着紗布的膠圈,將密封的膠圈輕輕一掀,就可看見藏在裏面的紅綢帶、白手帕和火山石打磨出的項鍊等物件。不用說,這是他爲變戲法而設置的一道機關,是他的魔法的後花園。

獨臂人對雲領說,阿姨不是魔術師,這下你死了心了吧?天晚了,阿姨該歇着了,咱回家吧。

雲領答應着,將草帽扣回頭上。我將梳妝檯上的錢拿起,還給獨臂人,他有些不好意思地接了,攥在手心中,說,明兒你去我那兒再選幾塊磨腳石,帶回城裏送人去吧。

我對獨臂人說不必了。我轉向雲領,請求他七月十五放河燈時將我也帶上。雲領看了看父親,又看了看我,最後盯着自己的鞋尖又看了半晌,纔對我說,你要是給你家魔術師放河燈,我就帶着你。我說當然了,我不會給別人放河燈的。雲領又說,你別穿高跟鞋,路很遠。我點了點頭。雲領就對父親說,那你今年得多做一盞河燈了。

七月十五的夜晚,我早早就吃過飯,換上旅遊鞋在房間裏等雲領。站在窗前,可望見升騰着的焰火。焰火是人世間最短暫又最光華的生命,欣賞它的輝煌時,就免不了爲它瞬間的寂滅而哀嘆。七點左右,雲領來了,他仍然穿着藏藍色的衣服,不過沒戴草帽,這使他看上去顯得高了一些。他挎着一隻腰鼓形的竹籃,籃子上放着一束紫色的野菊花。我想河燈一定掩映在野菊花下。

月亮已經走了一程路了,它彷彿是經過了天河之水的淘洗,光潤而明媚。我跟着雲領走出三山湖景區,踏上一條小路。

明月中的黑夜就不是真正的黑夜了,不僅小路清晰得像一條閃着銀光的緞帶,就連路邊矮樹叢中的各種形態的樹葉也能看得清楚。我問雲領要走多遠,他說到了地方你就知道多遠了。我又問他,你爸的胳膊是怎麼沒了的?雲領說,他不是在景區給遊人放焰火麼,我媽走了的第二年,有一個南方來的老闆非讓我爸手託着大禮花給他放,那天是那個老闆的生日。禮花有一個紙箱那麼大,值一千多塊錢呢。我爸幫他放這個禮花,他給二百塊錢。哪知道這禮花跟炸藥包一樣勁大,一點着火就把我爸掀了個跟頭,焰火上天了,我爸的一條胳膊也跟着上天了。從那以後,他才帶着我賣火山石的。

我嘆息了一聲,聽着雲領的腳步聲,看着月光裹挾着的這個經歷了生活之痛的小小身影,驀然想起蔣百嫂家那個轟鳴着的冰櫃,想起蔣三生,我突然覺得自己所經歷的生活變故是那麼那麼的輕,輕得就像月亮旁絲絲縷縷的浮雲。

穿過一片茂密的樹叢後,雲領問我聽到什麼沒有?我停下來,諦聽片刻,先聞幾聲鳥語,接着便是淙淙的水聲。雲領對我說,清流到了。

據云領講,清流是離三山湖最遠、也是最清澈的一條小溪。他媽媽曾對他講,一個人要是丟了,只要到清流來,喚幾聲他的名字,他的魂靈就會回來。

月光下的清流蜿蜒曲折,水聲潺潺。這條一腳就能跨過去的小溪就像固定在大地的一根琴絃。彈撥它的,是清風、月光以及一雙少年的手。雲領放下籃子,撩開野菊花,取出兩盞河燈,又取出火柴,一一將它們點燃,將一盞蓮花形的送給我。他對我說,他媽媽喜歡吃南瓜,所以他每年放的河燈都是南瓜形的。雲領先把幾枝野菊花放在清流上,然後怕我攪擾了他似的,捧着河燈去了上游。我打量着那盞屬於魔術師的蓮花形的河燈,它用明黃色的油紙做成,燭光將它映得晶瑩剔透。我從隨身的包中取出魔術師的剃鬚刀盒,打開漆黑的外殼,從中取出閃着銀光的剃鬚刀,摳開後蓋,將槽中那些細若塵埃的鬍鬚輕輕傾入河燈中。我不想再讓浸透着他血液的鬍鬚囚禁在一個黑盒子中,囚禁在我的懷念中,讓它們隨着清流而去吧。我呼喚着魔術師的名字,將河燈捧入水中。它一入水先是在一個小小的旋渦處聳了聳身子,彷彿在與我做最後的告別,之後便悠然向下遊漂盪而去。我將剃鬚刀放回原處,合上漆黑的外殼。雖然那裏是沒有光明的,但我覺得它不再是虛空和黑暗的,清流的月光和清風一定在裏面盪漾着。我的心裏不再有那種被遺棄的委屈和哀痛,在這個夜晚,天與地完美地銜接到了一起,我確信這清流上的河燈可以一路走到銀河之中。

從清流返回的路上,我和雲領都沒有講話。月亮因爲升得高了,看上去似乎小了一些,但它的光華卻是越來越動人了。我們才進三山湖景區,就望見獨臂人像棵漆黑的椴樹一樣,候在月光下。我謝過這對父子,回到旅館,換下旅遊鞋,清清爽爽地洗了個澡,將裝着剃鬚刀的盒子放在牀頭櫃上,半倚牀頭,回味着這次旅行。突然,我聽見盒子發出撲簌簌的聲音,像風一樣,好像誰在裏面竊竊私語着,這讓我吃驚不已。然而這聲音只是響了一刻,很快就消失了。不過沒隔多久,撲簌簌的聲音再次傳來,我便將那個盒子打開,竟然是一隻蝴蝶,它像精靈一樣從裏面飛旋而出!它扇動着湖藍色的翅膀,悠然地環繞着我轉了一圈,然後無聲地落在我右手的無名指上,彷彿要爲我戴上一枚藍寶石的戒指。

\end{document}
